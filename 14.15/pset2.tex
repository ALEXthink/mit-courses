\documentclass[psamsfonts]{amsart}

%-------Packages---------
\usepackage{amssymb,amsfonts}
\usepackage{enumerate}
\usepackage[margin=1in]{geometry}
\usepackage{amsthm}
\usepackage{theorem}
\usepackage{verbatim}
\usepackage{graphics}
\usepackage{float}

\newenvironment{sol}{{\bfseries Solution:}}{\qedsymbol}
\newenvironment{prob}{{\bfseries Problem:}}

\bibliographystyle{plain}

\voffset = -10pt
\headheight = 0pt
\topmargin = -20pt
\textheight = 690pt

%--------Meta Data: Fill in your info------
\title{14.15 \\
Networks \\
Problem Set 2}

\author{John Wang}

\begin{document}

\maketitle

Collaborators: Ryan Liu

\section{Problem 1}

\begin{prob}
  Let the directed graph $G$ be a ring: node $i$ is connected to $i + 1$ if $i < m$ and $m$ is connected to 1. Compute both eigenvalue centrality and Katz centrality (with $\beta = 1$). Comment on your result.

  Do the same for a $k$-regular undirected network (i.e., an undirected network in which every vertex has degree $k$). You may find the steps outlined in Newman Problem 7.1 helpful. Comment on your result.
\end{prob}
\begin{sol}
For the directed ring, there are edges from nodes $i$ to $i+1$, except when $i = m$, in which case there is an edge from $m$ to 1. Thus, the adjacency matrix looks like the following:
\begin{eqnarray}
  A^T = \left[
  \begin{array}{c c c c c}
    0 & 1 & 0 & 0 & 0 \\
    0 & 0 & 1 & 0 & 0 \\
    0 & 0 & 0 & 1 & 0 \\
    0 & 0 & 0 & 0 & 1 \\
    1 & 0 & 0 & 0 & 0
  \end{array}
  \right]
\end{eqnarray}

There is a set of ones above the diagonal, and a one in the bottom left hand side of the matrix.

To compute the eigenvalue centrality, we first need to compute the largest eigenvalue. However, we note that the adjacency matrix is doubly stochastic and therefore that the largest eigenvalue is 1. To compute the eigenvalue centrality, we have the following formula:
\begin{eqnarray}
  x_i = \frac{1}{\lambda} \sum_{j \in M(i)} x_j
\end{eqnarray}

Since we know that $\sum_{j \in M(i)} x_j = 1$, we see that $x_j = 1$ for all nodes. Therefore the eigenvalue centrality is 1 for all nodes.

As for Katz centrality for a ring graph, we have $x_{i}(k+1) = \alpha x_j + 1$. We know by symmetry that all nodes should have the same Katz centrality. Therefore, we see that $x_{i} = x_{j}$ so that $x_i = \frac{1}{1-\alpha}$. Thus, we find that the Katz centrality is $\frac{1}{1- \alpha}$.

Now let us move on to the $k$ regular graph. First, we notice that if every vertex has degree $k$, then each column in the adjacency matrix sums to $k$. This means that if we divide the entire matrix by $k$, then each column sums to 1 and we have a stochastic matrix. This implies that the largest eigenvalue is $1$. If we multiply this matrix by $k$ and obtain our original adjacency matrix, we see that the largest eigenvalue is $k$. Therefore, we can obtain our eigenvalue centrality using symmetry again so that $x_i = x_j \forall j$ which implies $x_v = \frac{1}{k}$. Therefore, we see that the eigenvalue centrality is $\frac{1}{k}$.

For Katz centrality of the $k$ regular graph, we know that $x_i = \alpha \sum_{j} A_{ij} x_j + 1$. This becomes $x_i = \alpha k x_j + 1$ by symmetry so that we have $x_i (1 - \alpha k) = 1$. Therefore, we find that the Katz centrality is $\frac{1}{1 - \alpha k}$.

The result is interesting because one can think of the ring as a $1$ regular graph. We see that if we set $k=1$, then the formulas for the Katz and eigenvalue centralities of both graphs agree.
\end{sol}

\section{Problem 2}

\begin{prob}
Suppose a directed network takes the form of a tree with all edges pointing inward towards a central vertex. What is the PageRank centrality of the central vertex in terms of the single parameter $\alpha$ appearing in the definition of PageRank and the geodesic distances $d_i$ from each vertex $i$ to the central vertex?
\end{prob}
\begin{sol}
  We know that the PageRank centrality is given by the following formula:
  \begin{eqnarray}
    x_{i} = \alpha \sum_{j} A_{ij} \frac{x_j}{L(j)} + \beta
  \end{eqnarray}

  Here $L(j)$ is the out degree of node $j$. We see that in tree example, each node (except for the center node) has an out degree of $1$ so that $L(j) = 1$. Thus, our expression becomes $x_{i} = \alpha \sum_{j} A_{ij} x_j + \beta$.

  Now consider a particular node which is not the center node. We will figure out how much it contributes to the center node's PageRank centrality. The node starts with $\beta$ of contribution. Then, when it progresses to it's parent, the parent has a contribution of $\alpha (2 \beta) + \beta$ of which $\alpha \beta$ is the contribution of the original node. Continuing onwards, we can see that the contribution of the original node is $\alpha^{L} \beta$, where $L$ is the distance from the center node. However, this is just given by $L = d_i$.

  We can sum over all nodes $i$ and obtain the PageRank for the center node:
  \begin{eqnarray}
    \beta \sum_{i=1}^n \alpha^{d_i}
  \end{eqnarray}
\end{sol}

\section{Problem 3}

\subsection{Problem 3.1}

\begin{prob}
Give an example of such a graph with 5 nodes.
\end{prob}
\begin{sol}
  I will use the example of a broken ring graph's adjacency matrix. This matrix is the following:
\begin{eqnarray}
  \left[
  \begin{array}{c c c c c}
    0 & 1 & 0 & 0 & 0 \\
    0 & 0 & 1 & 0 & 0 \\
    0 & 0 & 0 & 1 & 0 \\
    0 & 0 & 0 & 0 & 1 \\
    0 & 0 & 0 & 0 & 0
  \end{array}
  \right]
\end{eqnarray}

We can show it is nilpotent by taking the $A^5$ with gives the zero matrix. Intuitively, this can be verified by noting that $A^5$ gives all the possible routes of path $5$ from one node to another. Since this ring is broken, there are no such routes.
\end{sol}

\subsection{Problem 3.2}

\begin{prob}
  Compute the eigenvalue centrality. Explain your answer.
\end{prob}
\begin{sol}
  All eigenvalues of this matrix are 0. To see how this is the case, we notice that $A^k \to 0$ as $k \to \infty$ by the fact that $A^5 = 0$ so that $A^{5 + j} = 0 \forall j > 0$. By the decomposition of the expression of $A^k x = v_1 \lambda_1 + \ldots v_5 \lambda_5$, we see that all eigenvalues must be 0.

  We must therefore look at the left eigenvector associated with an eigenvalue of $0$. By inspection, we see that this eigenvector is $(1, 0, 0, 0, 0)$, so that the eigenvalue centrality of the first node is $1$ while the centrality of all the other nodes is $0$.
\end{sol}

\subsection{Problem 3.3}

\begin{prob}
  Compute the Katz centrality with $\beta = 1$. Explain your answer.
\end{prob}
\begin{sol}

\end{sol}

\section{Problem 4}

\begin{prob}
\end{prob}
\begin{sol}
\end{sol}

\begin{prob}
\end{prob}
\begin{sol}
\end{sol}

\begin{prob}
\end{prob}
\begin{sol}
\end{sol}

\section{Problem 5}

\begin{prob}
\end{prob}
\begin{sol}
\end{sol}
\begin{prob}
\end{prob}
\begin{sol}
\end{sol}

\section{Problem 6}

\begin{prob}
  Following the notation used in lecture, consider the MC on an undirected graph with adjacency matrix $A$ given by the transition matrix $AD^{-1}$. Assume that a steady state distribution  exists. Give an explicit expression of $\pi$ in terms of $A$ and $D$.
\end{prob}
\begin{sol}
The stationary distribution must satisfy the following $\pi = AD^{-1} \pi$ by definition. We will show that $\pi = \frac{diag(D^{-1})}{\sum_{j \in M} D_jj}$ is the stationary distribution where $diag(D^{-1})$ is the diagonal of $D^{-1}$ converted into a column vector.

If I show that this satisfies $\pi = AD^{-1} \pi$, then it is a valid stationary distribution. First, since matrices are association, we can consider $A(D^{-1} \pi)$:
\begin{eqnarray}
D^{-1}\pi = \left[
  \begin{array}{c c c}
    \frac{1}{d_1} & 0 & 0 \\
    0 & \frac{1}{d_2} & 0 \\
    0 & 0 & \frac{1}{d_3}
  \end{array}
\right]
\left[
  \begin{array}{c}
    \frac{d_1}{\sum_i d_i} \\
    \frac{d_2}{\sum_i d_i} \\
    \frac{d_3}{\sum_i d_i} \\
  \end{array}
\right] =
\left[
  \begin{array}{c}
    \frac{1}{\sum_{i} d_i} \\
    \frac{1}{\sum_{i} d_i} \\
    \frac{1}{\sum_{i} d_i} \\
  \end{array}
\right]
\end{eqnarray}

Notice that $D^{-1}\pi$ is just a vector of reciprocal of the sum of all degrees. Thus, when one takes this matrix and multiplies by $A$ to get $A (D^{-1} \pi)$. In this multiplication, notice that the adjacency matrix $A$ will be multiplied by $\frac{1}{\sum_i d_i}$ by the number of degrees that of each node to obtain the $i$th row in the resulting vector. This is just $\frac{d_i}{\sum_j d_j}$. However, this is exactly the resulting vector $\pi$, which shows that we have found a stationary distribution.
\end{sol}

\section{Problem 7}

\subsection{Problem 7.1}

\begin{prob}
  Show that the transition probability to $s$ from node $i$ is given by $p_{is} = \sum_{j \in S} p_{ij}$.
\end{prob}
\begin{sol}
  Consider all possible ways to enter $S$ from node $i$. These are only given by paths which connect $i$ to some node $j \in S$. Therefore, the total probability of entering $S$ is just the sum of the probabilities of following those paths, which is $\sum_{j \in S} p_{ij}$.
\end{sol}

\subsection{Problem 7.2}

\begin{prob}
  Show that $\mu = 1 + P_s \mu$ were $P_s$ is the matrix $\{ p_{ij} \}_1^n$.
\end{prob}
\begin{sol}
  We note that there is a simple recursive formula for each $\mu_i$ which depends on each $\mu_j$ for $j \neq i$, and the probability of leaving on each path to $s$. However, leaving on that path increases the path length by 1. Thus, we have the formula:
  \begin{eqnarray}
    \mu_i = 1 + \sum_{j \in S} \mu_j p_{ij}
  \end{eqnarray}

  The formula $\mu = 1 + P_s \mu$ follows by putting this in matrix form.
\end{sol}

\subsection{Problem 7.3}

\begin{prob}
  Show that $p_{is} = 1 - \sum_{j=1}^n p_{ij}$.
\end{prob}
\begin{sol}
  We know that $p_{is} = \sum_{j \in S} p_{ij}$ from problem 1. However, we know that $j \in S$ implies $j \notin n$. Therefore, we can rewrite $p_{is} = \sum_{j \notin n} p_{ij}$. Yet we know that the following is true by the fact that probabilities sum to 1:
  \begin{eqnarray}
    \sum_{j \in n} p_{ij} + \sum_{j \notin n} p_{ij} = 1
  \end{eqnarray}

  Thus, we can rearrange and solve for $p_{is}$ and we find that $p_{is} = 1 - \sum_{j \in n} p_{ij}$.
\end{sol}

\subsection{Problem 7.4}

\begin{prob}
  Let $1' P_s = (\hat{P}_1, \ldots, \hat{P}_n)$. Show that $0 \leq \hat{P}_i \leq 1$.
\end{prob}
\begin{sol}
  Notice that $\hat{P}_i$ is just the column sum of the $i$th column in $P_s$. Since $P$ is doubly stochastic we know that $\sum_{j \in s} p_{ij} + \sum_{j \in n} p_{ij} = 1$. Since $p_{ij} \geq 0$, we know that $\sum_{j \in s} p_{ij} \leq 1$. Moreover, this shows us that $\sum_{j \in s} p_{ij} \geq 0$. Since $\hat{P}_i = \sum_{j \in s} p_{ij}$, then we see that  $0 \leq \hat{P}_i \leq 1$.
\end{sol}

\subsection{Problem 7.5}

\begin{prob}
Show that $n = (1 - \hat{P}_1, \ldots, 1 - \hat{P}_n)\mu$.
\end{prob}
\begin{sol}
  Expanding out the right hand side of the above expression, we obtain:
  \begin{eqnarray}
    (1 - \hat{P}_1, \ldots, 1 - \hat{P}_n)\mu &=& \sum_{i=1}^n \mu_i - \sum_{i=1}^n \hat{P}_i \mu_i \\
                                              &=& \sum_{i=1}^n \mu_i - 1' P_S \sum_{i=1}^n \mu_i \\
                                              &=& \sum_{i=1}^n \mu_i - 1' (\mu - 1) \\
                                              &=& n
  \end{eqnarray}

  The second expression comes from problem 4, while the third expression comes from rearranging problem 2: $\mu - 1 = P_s \mu$. Finally, we obtain $n$ because $1' \mu = \sum_{i=1}^n \mu_i$ and because $1' 1 = n$.
\end{sol}

\subsection{Problem 7.6}

\begin{prob}
  Show that $n \leq (\max_j \mu_j) (1 - \hat{P}_1, \ldots, 1 - \hat{P}_n) 1$.
\end{prob}
\begin{sol}
  We know that $\mu_i \leq \max_j \mu_j$ for all $i$. Therefore, we see that
  \begin{eqnarray}
    n = (1 - \hat{P}_1, \ldots, 1 - \hat{P}_n) \mu \leq (1 - \hat{P}_1, \ldots, 1 - \hat{P}_n) 1 (\max_j \mu_j)
  \end{eqnarray}
\end{sol}

\subsection{Problem 7.7}

\begin{prob}
  Verify that $\frac{1}{n}(1 - \hat{P}_1, \ldots, 1 - \hat{P}_n) = \frac{1}{n} \sum_{i=1}^n p_{is} = \Phi(S)$.
\end{prob}
\begin{sol}
  We know that $\hat{P}_i$ is the sum of columns of $P_s$ so that $1 - \hat{P}_i = 1 - \sum_{j=1}^n p_{ij} = 1 - p_{is}$. Therefore, we see that the left hand side can be rewritten as:
  \begin{eqnarray}
    \frac{1}{n} ( 1 - \hat{P}_1, \ldots, 1 - \hat{P}_n) 1 &=& \frac{1}{n}(p_{1s}, \ldots, p_{ns}) 1 \\
                                                          &=& \frac{1}{n} \sum_{i=1}^n p_{is} \\
                                                          &=& \Phi(s)
  \end{eqnarray}
\end{sol}

\subsection{Problem 7.8}

\begin{prob}
  Conclude that $(\max_j \mu_j) \geq \frac{1}{\Phi(s)}$. 
\end{prob}
\begin{sol}
  We know from 6 that we have $\frac{1}{(\max_j \mu_j)} \leq \frac{(1 - \hat{P}_1, \ldots, 1 - \hat{P}_n)}{n} = \Phi(s)$. Thus, if we divide both sides by $\Phi(s)$ we obtain:
  \begin{eqnarray}
    (\max_j \mu_j) \geq \frac{1}{\Phi(s)}.
  \end{eqnarray}
\end{sol}
\end{document}
