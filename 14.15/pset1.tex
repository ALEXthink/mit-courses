\documentclass[psamsfonts]{amsart}

%-------Packages---------
\usepackage{amssymb,amsfonts}
\usepackage[all,arc]{xy}
\usepackage{enumerate}
\usepackage[margin=1in]{geometry}
\usepackage{amsthm}
\usepackage{theorem}
\usepackage{verbatim}
\usepackage{tikz}
\usetikzlibrary{shapes,arrows}

\newenvironment{sol}{{\bfseries Solution:}}{\qedsymbol}
\newenvironment{prob}{{\bfseries Problem:}}

\bibliographystyle{plain}

\voffset = -10pt
\headheight = 0pt
\topmargin = -20pt
\textheight = 690pt

%--------Meta Data: Fill in your info------
\title{14.15 \\
Networks \\
Problem Set 1}

\author{John Wang}

\begin{document}

\maketitle

Collaborators:

\section{Problem 1}

\begin{prob}
Consider an undirected tree of $n$ vertices. A particular edge in the tree joins vertices 1 and 2 and divides the tree into two disjoint regions of $n_1$ and $n_2$ vertices.

Show that the closeness centralities $C_1$ and $C_2$ of the two vertices, are related by
\begin{eqnarray}
  \frac{1}{C_1} + \frac{n_1}{n} = \frac{1}{C_2} + \frac{n_2}{n}.
\end{eqnarray}
\end{prob}
\begin{sol}
  First we notice that the sum of the distances of all shortest paths from $1$ $\sum_{j} d_{1j}$ has the following relation with $\sum_{j} d_{2j}$:
  \begin{eqnarray}
    \sum_{j} d_{1j} = \sum_{j} d_{2j} + n_2 - n_1
  \end{eqnarray}

  The above relation follows because the shortest distance from vertex 1 to all vertices in group $n_1$ is exactly one smaller than the shortest distance from vertex 2 to all vertices in group $n_1$. This follows because the only path to from vertex 2 to $n_1$ is through vertex 1 since that edge divides the tree into two disjoint regions. Moreover, by the same logic, the shortest distance from vertex 1 to group $n_2$ is exactly one larger than the shortest distance from vertex 2 to all vertices in group $n_2$. Using this logic, we see that $\sum_{j} d_{1j}$ is simply $\sum_{j} d_{2j}$ corrected for vertex 1 being one closer to all vertices in group $n_1$ and one further from all vertices in group $n_2$. Therefore, we have derived the above relation.

  Now, assuming that $n \neq 0$, we can obtain the following:
  \begin{eqnarray}
    \frac{\sum_{j} d_{1j}}{n} &=& \frac{\sum_{j} d_{2j} + (n_2 - n_1)}{n} \\
    \frac{1}{C_1} &=& \frac{1}{C_2} + \frac{n_2}{n} - \frac{n_1}{n} \\
    \frac{1}{C_1} + \frac{n_1}{n} &=& \frac{1}{C_2} + \frac{n_2}{n}
  \end{eqnarray}

  The final expression is what we wanted to show.
\end{sol}

\section{Problem 2}

Consider an undirected (connected) tree of $n$ vertices. Suppose that a particular vertex in the tree has degree $k$ so that its removal would divide the tree into $k$ disjoint regions, and suppose that the sizes of those regions are $n_1, \ldots, n_k$. 

\begin{prob}
  Show that the unnormalized betweenness centrality of $x$ of the vertex is $x = n^2 - \sum_{m=1}^k n^2_m$.
\end{prob}
\begin{sol}
First, notice that within each of the $k$ disjoint regions, the vertex $x$ does not fall on any shortest path between vertices $i$ and $j$, where $i,j$ are both in the same disjoint region. Without loss of generality, let us examine the $l$th disjoint region. Here, $x$ connects to some vertex $j$ in $l$. However, $j$ is strictly closer than $x$ to any other vertex $i$ in $l$. If this were not the case, then there must be some path from $x$ to $j$ which does not pass through $j$, but this is impossible because this path would lead to a closed loop (and there are no closed loops in trees).

It is obvious, then, that no shortest paths between two vertices in $l$ will pass through $x$. However, we know that the shortest path for vertex $i$ in one region and $j$ in a different region must pass through $x$ because $x$ is the only vertex which connects these two regions. Deleting $x$ causes the regions to become separated, which implies that $x$ is necessarily the only node connecting these regions and hence is in all shortest paths between these regions.

Thus we see that $x$ is contained in all shortest paths between two nodes $i$ and $j$ unless $i$ and $j$ are in the same region.

Notice that the number of shortest paths between $k$ nodes is $k^2$. It follows that the number of shortest paths which go through $x$ is the total possible number of shortest paths ($n^2$) minus the number of possible shortest paths through each region ($\sum_{m=1}^k n_m^2$). Thus, we find:

\begin{eqnarray}
x = n^2 - \sum_{m=1}^k n^2_m.
\end{eqnarray}
\end{sol}

\begin{prob}
Hence, or otherwise, calculate the betweeness of the $i$th vertex from the end of a line graph of $n$ vertices.
\end{prob}
\begin{sol}
We can apply the theorem that we proved above. Notice that the $i$th vertex has degree 2 and its removal would divide up the line (which is also a tree) into 2 distinct regions. The first region has $i-1$ nodes, while the second region has $n-i$ nodes. This means the betweenness of the $i$th vertex by the above theorem is
\begin{eqnarray}
n^2 - (i-1)^2 - (n-i)^2
\end{eqnarray}
\end{sol}

\end{document}
