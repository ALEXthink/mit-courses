\documentclass[psamsfonts]{amsart}

%-------Packages---------
\usepackage{amssymb,amsfonts}
\usepackage{enumerate}
\usepackage[margin=1in]{geometry}
\usepackage{amsthm}
\usepackage{theorem}
\usepackage{verbatim}
\usepackage{tikz}
\usetikzlibrary{shapes,arrows}

\newenvironment{sol}{{\bfseries Solution:}}{\qedsymbol}
\newenvironment{prob}{{\bfseries Problem:}}

\bibliographystyle{plain}

\voffset = -10pt
\headheight = 0pt
\topmargin = -20pt
\textheight = 690pt

%--------Meta Data: Fill in your info------
\title{Knewton Data Challenge}
\author{John Wang}

\begin{document}

\maketitle

\section{Problem Formulation}

Let there be $n$ individuals taking an exam each year, each of whom take $k$ of the $K$ total questions available. We require that $L$ of the $K$ questions be offered to at least one student, and also that $0 < k < K$ be satisfied. In this paper, I will examine strategies for finding a ranking of students given a previous year's results as training data.

\section{Measuring Question Difficulty}

In order to do this, we want to formulate some measure of the difficulty of each question $j$. This motivates our examination of $r_j$, the probability that a student will get question $j$ correct. To estimate $r_j$, one could naively use the sample mean from the training data for each question:
\begin{eqnarray}
r_j \approx \frac{1}{n_j} \sum_{i=1}^n x_{ij}
\end{eqnarray}

Where $x_{ij}$ denotes whether or not individual $i$ answered question $j$ correctly and $n_j$ is the number of times question $j$ was asked in the training data. This scheme works as long as the questions were assigned uniformly at random. However, as soon as there exists dependence among the questions, then this technique no longer works. If some set of questions $q_1$ are assigned to a group of students $s_1$ with higher probability than other questions and $s_1$ has a higher intelligence level than the average student, then $r_j$ for $j \in q_1$ will be biased upwards. 

Therefore, it is necessary to introduce a new variable $\theta_i$ which captures the intelligence level of student $i$. The probability that student $i$ answers question $j$ correctly will now have an additive factor of $\theta_i$ and will be given by $r_j + \theta_i$. If $r_j + \theta_i \geq 1$, then student $i$ always answers $j$ correctly, and if $r_j + \theta_i \leq 0$, then student $i$ never answers question $j$ correctly. Let $\gamma_{ij} = 1$ if student $i$ was assigned question $j$ and $\gamma_{ij} = 0$ otherwise. We will try to minimize the distance to the probability prediction:

\begin{eqnarray}
\sum_{i=1}^K | x_{ij} - (r_j + \theta_i)| \gamma_{ij}
\end{eqnarray}

Using this distance function, we will attempt to estimate $r_j$ for all $j$ and $\theta_i$ for all $i$ with the following algorithm. We start by initializing $\theta_i = 0$ for all $i$ and $r_j = \frac{1}{n_j} \sum_{i=1}^n x_{ij}$ for all $j$. Now will will loop for $T$ iterations. Suppose we are in iteration $t > 0$ and $t \leq T$. We will first update all intelligence values $\theta_i$ for all $i$ by estimating $\sum_{j=1}^K |x_{ij} - (r_j + \theta_i)| \gamma_{ij}$. We will also estimate this with $\theta_i$ replaced with $\theta_i + \Delta_t$ and $\theta_i - \Delta_t$. The algorithm then replaces $\theta_i$ with $\theta_i - \Delta_t, \theta_i, \theta_i + \Delta_t$ depending on which one gives the smallest distance. We do this for all $i$ and do the same for $r_j$ by calculating $\sum_{i=1}^n |x_{ij} - (r_j + \theta_i)| \gamma_{ij}$ for $r_j - \Delta_t, r_j, r_j + \Delta_t$. We do this for all $j$ and finish the iteration.

To calculate the distance offset $\Delta_t$ at each iteration $t$, we will use the function $\Delta_t = \frac{1}{2} e^{-t}$. This allows our distance to decrease at each iteration in an exponential manner. Intuitively this allows the algorithm to explore the sample space at the beginning and narrow down the search in later iterations.

To prove that the resulting $r_j$ and $\theta_i$ will be good approximations of their actual values, we will show that our distance to the optimal result is bounded. Let $r^*_j$ be the optimal value of $r_j$ and $\theta^*_i$ be the optimal value of $\theta_i$ for all $j$ and $i$. The distance using these values is given by:
\begin{eqnarray}
E \left[ \sum_{j=1}^K | x_{ij} - (r^*_i + \theta^*_i) | \gamma_{ij} \right] = round(k ( r^*_i + \theta^*_i)) - k ( r^*_i + \theta^*_i)
\end{eqnarray}

Here, $round(.)$ denotes the function that rounds $.$ to the nearest integer. The expression follows because there will be $k$ questions for which $\gamma_{ij} = 1$ for each student $i$, and since there can only be an integer number of $x_{ij} = 1$. 


\end{document}
