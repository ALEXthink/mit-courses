\documentclass[psamsfonts]{amsart}

%-------Packages---------
\usepackage{amssymb,amsfonts}
\usepackage[all,arc]{xy}
\usepackage{enumerate}
\usepackage[margin=1in]{geometry}
\usepackage{amsthm}
\usepackage{theorem}
\usepackage{verbatim}
\usepackage{framed}
\usepackage{tikz}
\usetikzlibrary{shapes,arrows}

\newenvironment{sol}{\vspace{0.25cm}{\large \bfseries Solution:}}{\qedsymbol}
\newenvironment{prob}[1]{\begin{framed}{\large \bfseries Problem #1:}}{\end{framed}}
\newcommand{\makenewtitle}{
    \begin{center}
    {\huge \bfseries 6.854 Advanced Algorithms} \\
    Problem Set 8\\
    \vspace{0.25cm}
    {\bfseries John Wang} \\
    Collaborators:  
    \end{center}
    \vspace{0.5cm}
}


\bibliographystyle{plain}

\voffset = -10pt
\headheight = 0pt
\topmargin = -20pt
\textheight = 690pt

\begin{document}

\makenewtitle

\begin{prob}{1-a}
Suppose the optimum diameter $d$ is known. Devise a greedy 2-approximation algorithm (an algorithm which gets $k$ clusters each of diameter at most $2d$). 
\end{prob}
\begin{sol}
We start and pick an arbitrary center point $x_1$. This will be center of cluster $c_1$. We will place all points within a distance of $d$ from $x_1$ into cluster $c_1$. In other words, for all points $y$ such that $d(x_1, y) \leq d$, we will set $y \in c_1$. Now, we will continue this procedure $k$ times. At the end, we will have formed $k$ clusters, each of diameter at most $2d$. 

I will now show the correctness of this 2-approximation. First, we know that each point $x$ will be inside a cluster $c$ of diameter $d$. Therefore, a cluster of diameter $2d$ centered at $x$ will completely contain cluster $c$ as a subset. This follows because any point in cluster $c$ must be the center of the larger cluster, which implies that in any orientation of the point, the smaller cluster will be contained in the larger cluster. This means that OPT's cluster $c$ will be contained in the approximation algorithm's cluster. It is clear by the triangle inequality that no point in each cluster is more than a distance of $2d$ away from another point in the same cluster.

From this, we see that each one of OPT's clusters $k_i$ will be wholely contained by the approximation algorithm's clusters $c_i$. Thus, each point will be accounted for once the approximation algorithm terminates. Therefore, we see that the approximation alogirthm will find $k$ clusters each of diameter at most $2d$ which will cover the entire space of points. Moreover, since we know that the optimum diameter is $d$, we see that this is a 2-approximation.
\end{sol}

\begin{prob}{1-b}
Consider an algorithm which ($k$ times) chooses as a center the point at maximum distance from all previously chosen centers, then assigns each point to the nearest center. By relating this algorithm to the previous algorithm, show that you get a 2-approximation.
\end{prob}
\begin{sol}
If the algorithm picks points which are at least $d$ apart from each other, then each center will belong to a different cluster $k_i$ in OPT. This follows because clusters in OPT have a diameter of $d$, which means that if two points $x$ and $y$ have a distance $d(x,y) > d$, then $x$ and $y$ cannot belong to the same cluster. Now, if all $k$ centers are $d$ distance apart from each other, then we have created $k$ clusters where no point in any of the clusters is further than $d$ from each of the centers. This means the maximum diameter of the clusters will be $2d$.

Thus, all that needs to be shown for a 2-approximation is that the algorithm will pick center points which are at least $d$ apart from each other. Suppose this is not true. Then after the algorithm has found $p$ centers that are $d$ apart from each other, there will no longer be any points which are $d$ apart from all the previously chosen centers. This, however, implies that all of the remaining points can be allocated into some of the clusters which already exist. This means that we do not need all $k$ of the centers for OPT's algorithm, which means we can decrease the maximum diameter (in the worst case, we can set the diameter to the second larger diameter, and make a cluster for the single point which is left out). This shows that OPT does not provide the optimal diameter, which is a contradiction. Therefore, we algorithm must pick center poitns which are at least $d$ apart from each other and so the algorithm is a 2-approximation.
\end{sol}

\newpage
\makenewtitle

\begin{prob}{2-a}
Argue that the greedy algorithm (repeatedly take any edge that does not conflict with previous choices) can be implemented in linear time and gives a 2-approximation to the maximum (number of edges) bipartite matching.
\end{prob}

\begin{sol}
First, we shall argue that repeatedly taking any edge that does not conflict with previous choices can be used to create an approximation algorithm that runs in linear time. The nodes are ordered arbitrarily, but they must be ordered so that edges originating from the same node are placed together. This ordering can be done in $O(n + m)$ time by using a list of lists for each node $u$, where all edges $(u,v)$ with the same starting node $u$ are grouped together. 

Next, we take the a single edge from each list of lists and include it in the bipartite matching, and remove the entire list from the overall list. Additionally, we keep a hash of all the nodes $v$ for which an edge $(u,v)$ is incident upon. If we select an edge to delete where either node $u$ or $v$ is already used, then we throw away that edge and continue. 

In total, we will examine no more than $m$ edges, so the total algorithm runs in $O(m+n)$, which is linear. Next, we note that this is a 2-approximation. 

Let us say that the edge that the approximation algorithm choose $(u_1, v_1)$ is not the same as the edge OPT chooses for $v_1$. Namely, suppose OPT choose $(u_0, v_1)$. Then in the worst case, OPT can choose a different edge $(u_1, v_2)$ starting from $u_1$. However, we see that every incorrect choice that the approximation algorithm makes only affects 2 choices that OPT makes. Thus, since the approximation algorithm is still creating a matching, OPT do at maximum twice as well as the approximation algorithm. This follows because for each edge $(u_1, v_1)$ selected by the approximation algorithm, OPT can only differ by two edges, which means OPT can only improve by twice as much. Thus, this algorithm is a 2-approximation.
\end{sol}

\begin{prob}{2-b}
Generalize to argue that when edges has positive weights, the greedy algorithm (consider edges in decreasing order of weight) can be implemented in $O(m \log n)$ time and is a 2-approximation algorithm for maximum (total) weight bipartite matching.
\end{prob}

\begin{sol}
It is clear that the algorithm can be implemented in $O(m \log n)$ time. For each node $u$ for which there is an outgoing edge $(u,v)$, take the highest weight edge of all outgoing edges. Sort all of these highest weight edges and put them into the matching. This is equivalent to examining the edges in decreasing order by weight because each source node $u$ can only provide a single edge in the matching, which means that edge must be the highest weight edge leaving $u$.  

To show that this algorithm is a 2-approximation, we make a similar argument as above. We note that whenever the approximation algorithm chooses an edges $(u_1, v_1)$, OPT could have chosen two different edges, one starting on $u_1$ and one ending at $v_1$. However, we know that both these two possible edges have weight less than the edge $(u_1, v_1)$, or else they would have been selected. This means that OPT can possibly have twice as large of total weight per edge that is selected by the approximation algorithm. It holds therefore, that this algorithm is a 2-approximation.
\end{sol}

\begin{prob}{2-c}
Show that the same holds for a general (non-bipartite) max-weight matching problem.
\end{prob}
\begin{sol}
The algorithm remains the same. We sort the edges in each node and select the highest weight edge form each node. We shall show that this algorithm still works on a general graph to create a 2-approximation.

Let $s$ be the weight of the first edge which is removed (thus, $s$ is the maximum edge weight on the graph). Let $s$ correspond to the weight on edge $(u,v)$. Then we know that when the algorithm removes edge $(u,v)$ and all incident edges on $u$ and $v$, then at most two edges from the optimal matching will be removed. This follows because there can be at most two edges in the optimal matching which are incident on $u$ and $v$.

However, we know that both of these edges must have weight less than or equal to $s$. Thus, the optimal matching for all edges incident on nodes $u$ and $v$ has weight less than or equal to $2s$. Recursively performing the same procedure for the other $n-2$ nodes shows that the optimal matching will have weight less than or equal to twice times the weight of the approximation algorithm. This means that we have provided a 2-approximation. Note that this proof only works when edge weights are non-negative.
\end{sol}

\newpage
\makenewtitle

\begin{prob}{3-a}
Suppose you compute all-pairs shortest paths in $G$, and create a complete graph $G'$ on the termainsl $T$ where each hedge cost is equal to the shortest path between its (terminal) endpoints in $G$. Relate the cost of the minimum spanning tree in this graph to the cost of the optimum steiner tree in $G$. 
\end{prob}
\begin{sol}
We see that the optimal Steiner tree in $G$ has a cost which is greater than half of the cost of the minimum spanning tree in $G'$. First, let us assume we have found the optimal Steiner tree $S$. Then, from this Steiner tree, we can construct a minimum spanning tree in $G'$. 

We perform a DFS and visit all of the nodes in $S$ starting from the root of the Steiner tree. The order in which the terminals in $T$ are visited will be the order we can use to construct a minimum spanning tree. Suppose we visit terminals $v_1, v_2, \ldots, v_n \in T$ in that order during our DFS of $S$. Then we see that we are always moving along a shortest path from $v_i$ to $v_{i+1}$ since Steiner trees guarantee that we have a minimum cost span across all vertices. If this were not the case, then we could find some shorter path between $v_i$ and $v_{i+1}$ and exchange that path in the Steiner tree, which would find a more optimal Steiner tree. Since we were assuming the optimality of our Steiner tree, this is a contradiction. 

Since we are always taking shortest paths between $v_{i}$ and $v_{i+1}$, we know that we have a minimum spanning tree in $G'$ between $v_{i} \forall i \in [1, n]$. Thus, our path in $S$ has led to a minimum spanning tree in $G'$. Now, we shall show that the DFS in $S$ traverses a path with with a cost at most twice of that of $S$. This is because the DFS visits each edge at most twice (once up and once down the DFS search). This shows that the cost of the spanning tree in $G$ is greater than half of the cost of the minimum spanning tree in $G'$. 
\end{sol}

\begin{prob}{3-b}
Give a 2-approximation for the Steiner tree problem.
\end{prob}
\begin{sol}
We will use the structure in $G'$ to create an approximation. We will first take any arbitrary terminal node $v_1$ and choose it as our starting point. Next, we will search in $G'$ for its lowest weight neighbor. In other words, we will look for the shortest of all the shortest paths from $v_1$ to other terminals. We will include this shortest path in the Steiner tree that we build. Once we have a partially built Steiner tree, we will look for a terminal $v_2$ which is closest to any node in the Steiner tree. We take the shortest path from any node $v_i$ in the Steiner tree to any other node $u_j$ not yet in the Steiner tree. Once we find this shortest path, we add it to the Steiner tree. We continue this until all terminal nodes are included in the Steiner tree.

This will be a 2-approximation because we know that the cost of the Steiner tree we have built will not be more than the cost of the optimal spanning tree in $G'$. Since we know that the optimal spanning tree in $G'$ is less than twice the cost of the optimal Steiner tree in $G$, we know that the cost of our Steiner tree will be less than twice the cost of an optimal Steiner tree.
\end{sol}

\newpage
\makenewtitle

\begin{prob}{4-a}
Prove that a two colorable graph can be colored with 2 colors in polynomial time.
\end{prob}

\begin{sol}
Let a graph $G$ be two colorable. We wish to color the vertices $v_i in G$ with two colors $c_0$ and $c_1$. Then we shall start at an arbitrary vertex $v_1$ and color it with $c_0$. Now, perform a BFS starting at vertex $v_1$. Each level of the BFS will be colored an alternating color. Thus, the first level of the BFS (which consists of all of the neighbors of $v_1$), will be colored $c_1$. The next level will be colored $c_0$, etc. Perform this for all disconnected components of $G$, and afterwards, we will have a 2 coloring.

To show that this provides a valid two-coloring, we will assume without loss of generality that $G$ is connected (otherwise we can break the problem up into smaller problems of connected graphs). Now, since $G$ has a valid two-coloring, we know that for any node $v$ with color $c_i$, the neighbors of $v$ must have color $c_j$ where $j \neq i$. This follows immediately because a two-coloring means that all neighbors must have the other color in the set of colors. This means that performing a BFS by levels (which enumerates all the neighbors of node $v$), we will be able to color neighbors in different colors. Moreover, since $G$ is two colorable, all of the vertices and their neighbors are automatically constrained, so that as soon as a single color in a node is placed, all of the other vertices will be constrained to have a certain color. This dimensionality reduction means that our BFS algorithm will work correctly.
\end{sol}

\begin{prob}{4-b}
Show that a graph where every vertex has degree at most $d$ can be colored with $d+1$ colors using the obvious greedy algorithm.
\end{prob}

\begin{sol}
We will use the following greedy algorithm. For each vertex, we start out with a set of possible colors which includes all $d+1$ colors. Next, we start at some arbitrary vertex $v_0$ and pick a color $c_i$ from its set of possible colors $S(v_0)$. Once we have choosen $c_i$, we will remove $c_i$ from the sets $S(v_j)$ for all vertices $v_j$ which are neighbors of $v_0$. We continue this procedure for another vertex $v_1$, and keep going until we have finished picking colors for all vertices. This algorithm will find a valid $d+1$ coloring if it exists, because we can always be sure that we will not be violating any coloring conditions. Moreover, since the colors are arbitrary and no more than $d$ edges are incident upon any vertex, we will be sure that the set $S(v_i)$ will contain at least 1 item, because we can only remove a single item for each indicident edge.

To show that there exists a valid $d+1$ coloring, we will use induction on the number of vertices in the graph. First, we note that if a graph has less than or equal to $d+1$ vertices, then one can just choose a color for each vertex. Thus, we have shown the base case. Now we use our inductive hypothesis to assume that we can color a graph with $k$ vertices (whose maximum vertex has degree $d$) using $d+1$ colors. Take the inductive step and we must show this also holds for graphs with $k+1$ vertices. For any graph $G$ with $k+1$, we can take some arbitrary vertex $v$ and remove it and all its incident edges from the graph to create a new graph $G'$. Since $G'$ has $k$ vertices and still has a maximum vertex degree of $d$ (we didn't add any edges), we can use our induction hypothesis to find a $d+1$ coloring. Now, color all of the edges in $G'$ according to this coloring and add $v$ along with all its edges back into the graph. We know that $v$ has degree less than or equal to $d$, which means that there is at least one color which $v$ does not share with its neighbors. Pick one of these colors and set $v$'s color to it. 

Since we have colored $\{v \} \cup G'$ with $d+1$ colors (and we haven't violated any coloring constraints), we have found a valid coloring for a graph $G$ with $k+1$ vertices and degree at most $d$. Therefore, we have finished our induction step and have shown that all graphs with maximum degree $d$ can be colored with $d+1$ colors.
\end{sol}

\end{document}
