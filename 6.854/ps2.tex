\documentclass[psamsfonts]{amsart}

%-------Packages---------
\usepackage{amssymb,amsfonts}
\usepackage[all,arc]{xy}
\usepackage{enumerate}
\usepackage[margin=1in]{geometry}
\usepackage{amsthm}
\usepackage{theorem}
\usepackage{verbatim}
\usepackage{framed}
\usepackage{tikz}
\usetikzlibrary{shapes,arrows}

\newenvironment{sol}{\vspace{0.25cm}{\large \bfseries Solution:}}{\qedsymbol}

\newenvironment{prob}[1]{\begin{framed}{\large \bfseries Problem #1:}}{\end{framed}}

\newcommand{\makenewtitle}{
\begin{center}
{\huge \bfseries 6.854 Advanced Algorithms} \\
Problem Set 2\\
\vspace{0.25cm}
{\bfseries John Wang} \\
Collaborators:
\end{center}
\vspace{0.5cm}
}

\bibliographystyle{plain}

\voffset = -10pt
\headheight = 0pt
\topmargin = -20pt
\textheight = 690pt

%--------Meta Data: Fill in your info------
\begin{document}
\makenewtitle
\begin{prob}{1-a}
In class, I stated that single rotations don't work for splay trees. To demonstrate this, consider a degenerate $n$-node linked list shaped binary tree where each node's right child is empty. Suppose the only leaf is splayed to the root by single rotations: show the structure of the tree after this splay. Generalizing, argue that there is a sequence of $n/2$ splays that each take at least $n/2$ work.
\end{prob}

\begin{sol}
If we start with the given degenerate $n$-node tree and splay the bottom-left node (the leaf) $x$, then we will slowly move the node up to the root through a series of "kink" rotations. These kink rotations will move $x$ (which is the global minimum of the tree) and decrease its depth by one each time. Visually, all the nodes above and below $x$ will be left children of their parents, and $x$ will be the only node which is a right child of its parent. The process of moving $x$ up the tree will continue until $x$ becomes the root. At this point, $x$ will only have a right child, which will be the maximum element of the data structure, and everything else will continue down in a series of left children.

Let us call this resulting data structure $T'$. When we splay the new leaf of the tree $x'$ (node which can be found by walking down the left half of the tree), we can see that it will take the same path up to the root through a series of "kink" rotations. Once $x'$ becomes a right child of the root $x$, a rotation will occur which puts $x'$ at the root of the tree, and $x$ as the left child (since $x$ is the absolute minimum and is the only node smaller than $x'$). The right child of $x'$ will be the global maximum. The tree now has depth $n-1$.

Continuing in the same pattern and taking the deepest leaf $x''$ of the tree, we will go through a series of $n-2$ rotations until $x''$ reaches the root. At this point $x'$ will be the left child of $x''$, with $x$ as the left child of $x'$. The maximum node will be the right child of the new root, so the tree will now have depth $n-2$. 

It is clear that the deepest leaf $a$ will always be promoted up to the leaf in time equal to the depth of the tree. Once it has reached the root, the previous root $b$ will become its left child, and the maximum node which was previously $b$'s right child, will become $a$'s right child. Therefore, the tree will decrease in depth by $1$. 

A series of $n/2$ operations that splay the deepest leaf will therefore cost $O(d)$ each, where $d$ ranges from $n, \ldots, n/2$. Therefore, we have shown that there is a sequence of $n/2$ splays that each take at least $n/2$ work.
\end{sol}

\begin{prob}{1-b}
Now from the same starting tree, show the final structure after splaying the leaf with (zig-zig) double rotations. Explain how this splay has made much more progress than single rotations in improving the tree.
\end{prob}

\begin{sol}
Starting the the only leaf of the tree $x$, we will perform zig-zigs on the way up to the root. For notation, we will say $x$ is the leaf (minimum element), $x'$ is the parent of the leaf (second smallest element), and $x''$ is the parent of $x'$, etc. After the first zig-zig, we see that everything except the last three nodes have been untouched, but $x$ is now the child of $x'''$. Moreover, $x$ has a right child of $x'$ and $x'$ has a right child of $x''$. 

Performing another zig-zig operation, the $x'$ to $x''$ chain will become the left subtree to $x'''$. Moreover, $x'''$ will have a right child of $x''''$. In order to see the generalized behavior, we will call $A$ the right subtree of $x$. Note that $x$ has no left child at all. Now, $x$ is the left child of its parent $y$, and $y$ is also a left child of its parent, $z$. Thus, performing a zig-zig operation will bring $x$ to the top of $y$ and $z$. Thus, $x$ will have a right child $y$ and no left child, while $y$ has a right child of $z$ and a left child of $A$. 

It is now clear that zig-zig operation will bring $x$ up higher in the tree while adding the chain of $y$ and $z$ to the right subtree of $x$. Additionally, $y$'s left child will now be $x$'s previous subtree. 

Once the splay of $x$ is finished, each node on the left-most path from the root will have a right child. Thus, we see that using the zig-zig operation, the splay will decrease the depth of the tree by a multiplicative factor of $2$. In the previous case, the depth after a single splay was $n-1$, while now the depth is $n/2$, which is clearly much improved.
\end{sol}

\begin{prob}{1-c}
Given the theorem about access time in splay trees, it is tempting to conjecture that splaying does not create trees in which it would take a long time to find an item. Show that this conjecture is false by showing that for large enough $n$, it is possible to restructure any binary tree on $n$ nodes into any other binary tree on $n$ nodes by a sequence of splay operations. Conclude that it is possible to make a sequence of requests that cause the splay tree to achieve any desired shape.
\end{prob}

\begin{sol}
We will start by showing how we can use splay operations to make a specified node into a leaf. First, let us take an arbitrary node $x$ which is in some tree $T$. To make $x$ descend down the tree, we will splay the left and right child of $x$. Doing these splays will decrease the depth of the children, but increase the depth of $x$, sending it closer towards being a leaf. We perform this recursively until $x$ has no more children. At this point, $x$ has become a leaf. 

Now, assume we have tree $T$ which we want to restructure into $T'$. We first will find all the leaves of $T'$ using BFS. Then, we will perform a series of recursive splay operations on the nodes in $T$ which correspond to the leaves of $T'$ to make them leaves. Notice that once a node $y$ has become a leaf, no splay operation will be able to change $y$ into an internal node unless we specifically splay $y$. Therefore, it is possible to reorganize $T$ such that $T$ has the same leaves as $T'$. 

Once $T$ has the same leaves as $T'$, we must reorganize the internal nodes of $T$. 

\end{sol}

\newpage

\makenewtitle

\end{document}
