\documentclass[psamsfonts]{amsart}

%-------Packages---------
\usepackage{amssymb,amsfonts}
\usepackage{enumerate}
\usepackage[margin=1in]{geometry}
\usepackage{amsthm}
\usepackage{theorem}
\usepackage{verbatim}
\usepackage{framed}
\usepackage{tikz}
\usetikzlibrary{shapes,arrows}

\newenvironment{sol}{\vspace{0.25cm}{\large \bfseries Solution:}}{\qedsymbol}
\newenvironment{prob}[1]{\begin{framed}{\large \bfseries Problem #1:}}{\end{framed}}
\newcommand{\makenewtitle}{
    \begin{center}
    {\huge \bfseries 6.854 Advanced Algorithms} \\
    Problem Set 10\\
    \vspace{0.25cm}
    {\bfseries John Wang} \\
    Collaborators:  
    \end{center}
    \vspace{0.5cm}
}


\bibliographystyle{plain}

\voffset = -10pt
\headheight = 0pt
\topmargin = -20pt
\textheight = 690pt

\begin{document}

\makenewtitle

\begin{prob}{1-a}
Prove that LRU and FIFO are conservative.
\end{prob}
\begin{sol}
First, we will consider LRU. Suppose that the LRU cache of size $k$ has been filled up and is entering a new phase. Now consider a subsequence that contains at most $k$ pages. Consider pages $p_1, p_2, \ldots, p_k$ and LRU will check each whether $p_i$ is already in the cache. If it is, then there is a cache hit. Otherwise, there is a fault. However, since there are at most $k$ pages in the input sequence, there can be at most $k$ faults. Therefore, we see that LRU is conservative.

Now, let us examine FIFO with the same analysis. Consider a subsequence of pages $p_1, p_2, \ldots, p_k$. We know that there are exactly $k$ pages in the input sequence. Since each page in the sequence can cause at most a single fault, there can be a maximum of $k$ faults on FIFO. THerefore, we see that FIFO is conservative as well.
\end{sol}

\begin{prob}{1-b}
Prove that any conservative algorithm is $k$-competitive
\end{prob}
\begin{sol}
Suppose that some algorithm $A$ is conservative. We will break up the conservative algorithm's operation into phases of size $k$. Now suppose that the algorithm has just started phase $i$. In this phase, there will be $k$ page requests. Since the algorithm is conservative, $A$, will have a maximum of $k$ faults in this upcoming phase (which is really a subsequence of $k$ pages). Now let us suppose OPT does not fault in phase $i$. Then it must fault on the $k+1$st request (first request in the next phase) since the cache is only of size $k$ and it must have all $k$ requests already stored in the cache. If OPT does fault during phase $i$, then it has at least one fault. Therefore, for each phase $i$, there will be at least one fault by OPT and at most $k$ faults by $A$. 

Since this is true for each phase $i$, we see that $A$ is $k$-competitive, and hence, that any conservative algorithm is $k$-competitive.
\end{sol}
\end{document}
