\documentclass[psamsfonts]{amsart}

%-------Packages---------
\usepackage{amssymb,amsfonts}
\usepackage[all,arc]{xy}
\usepackage{enumerate}
\usepackage[margin=1in]{geometry}
\usepackage{amsthm}
\usepackage{theorem}
\usepackage{verbatim}
\usepackage{tikz}
\usetikzlibrary{shapes,arrows}

\newenvironment{sol}{{\bfseries Solution:}}{\qedsymbol}
\newenvironment{prob}{{\bfseries Problem:}}

\bibliographystyle{plain}

\voffset = -10pt
\headheight = 0pt
\topmargin = -20pt
\textheight = 690pt

%--------Meta Data: Fill in your info------
\title{6.854 \\
Advanced Algorithms \\
Problem Set 1}

\author{John Wang}

\begin{document}

\maketitle

Collaborators: Jason Hoch, Ryan Liu, Varun Ganesan

\section{Problem 1}

\begin{prob}
Unlike regular heaps, Fibonacci heaps do not achieve their good performance by keeping the depth of the heap small. Demonstrate this by exhibiting a sequence of Fibonacci heap operations on $n$ items that produce a heap-ordered tree of depth $\Omega(n)$.
\end{prob}

\begin{sol}
Consider the following recursive series of operations. For the $i$th step of the recursion, we will assume there is a tree $t_1$ of depth $i$, composed of exactly $i$ nodes. There is also a tree $t_2$ which is a single node such that $root(t_1) < root(t_2)$. We shall insert two nodes $a$ and $b$ such that $b < a < root(t_1) < root(t_2)$. Perform a delete-min operation on the set of trees. This operation will remove $b$ since it is the minimum of the entire structure. This leaves us with 3 roots, namely $a, root(t_1),$ and $root(t_2)$. The delete-min operation will also perform a consolidation, so that $a$ is merged with $t_1$, then the resulting tree is merged with $t_2$. 

The resulting tree has a root of $a$, a left child of $root(t_1)$ and a right child of $root(t_2)$. Now, we perform a decrease key on $root(t_2)$ to a value lower than $a$. This will cut it off from the tree, and we will be left with a tree $t'_1$ rooted at $a$ and $t'_2$. We see that $t'_1$ will have a depth of $i+1$ and $t'_2$ will be a single node. Thus, we are back to our original datastructure with step $i+1$ and can recurse. 

Note that we performed four operations: insert $a$, insert $b$, delete-min, decrease-key. Thus, we see that after $n$ of these operations, we will have a tree of length $n/4 = \Omega(n)$. 
\end{sol}

\section{Problem 2}

\begin{prob}
Suppose that Fibonacci heaps were modified so that a node was cut only after losing $k$ children. Show that this will improve the amortized cost of decrease key (to a better constant) at the cost of a worse cost for delete-min (by a constant factor).
\end{prob}

\begin{sol}
First we will define our potential function as $\Phi = R + 2M/(k-1)$ where $R$ is the number of roots and $M$ is the number of mark bits. Examining the amortized cost of insert $a_i$ is given by:
\begin{eqnarray}
a_i = c + 1 + \Delta \Phi
\end{eqnarray}

Where $c$ is the number of nodes cut on a given insert (due to cascading). The real cost is $c+1$ because $c$ nodes are cut during cascading, each requiring constant time, and $+1$ because the node must be inserted into the data structure as well. The change is potential is given by:
\begin{eqnarray}
\Delta \Phi = c + \frac{2(1 -(k-1)(c-1))}{k-1}
\end{eqnarray}

Because $c$ nodes are cut during the cascading, an additional $c$ roots are created which accounts for the first $c$ term in $\Delta \Phi$. Moreover, the number of mark bits decreases by $(k-1)(c-1)$ since we cut away $c$ nodes, which means that $c-1$ of these nodes had $k-1$ mark bits already stored which were cleared when everything was cascaded. However, we added $1$ mark bit to the last node in the cascading chain, which is why we have a change of $1 - (k-1)(c-1)$ mark bits. Putting this into our expression, we obtain:
\begin{eqnarray}
a_i &=& 1 + c + c + \frac{2(1 -(k-1)(c-1))}{k-1}\\
&=& 1 + 2 + \frac{2}{k-1}\\
&=& 3 + \frac{2}{k-1}
\end{eqnarray}

Thus, when $k=2$, the cost for insert is $5$, whereas when $k > 2$, the cost for insert is less than $5$. Thus, the change improves the amortized cost of decrease key to a better constant.

We are left to show that cutting nodes only after losing $k$ children makes delete-min more expensive. We know that the real cost of delete min will be the number of roots $r$ plus the maximum degree possible for the root. Since the potential function is only changed by the number of roots, we see that $\Delta \Phi = \text{max degree} - r$. Therefore, we see that the amortized cost is simply $a_i = O(\text{ \# of roots }$.

Now to find the maximum degree, we need to consider node $x$ with children $y_0, y_1, \ldots, y_i$ in the order they were added. We know $y_i$ becomes a child of $x$ during consolidation and the degree of $x$ is at least $i - 1$ because $y_0, \ldots, y_{i-1}$ were already children of $x$. This means that $y_i$'s degree could only have decreased by $k-1$ before it gets cut so that $deg(y_i) \geq i-1 - (k-1) = i-k$. This means we can find a lower bound for the maximum degree, using $S_m$ where $S_m$ is the minimum number of nodes in a degree $m$ subtree. This means:
\begin{eqnarray}
S_m &\geq& \sum_{i=k}^m S_{i-k} \\
S_m - S_{m-1} &\geq& \sum_{i=k}^m S_{i-k} - \sum_{i=k}^{m-1} S_{i-k} = S_{m-k}
\end{eqnarray}

Since $S_m - S_{m-1} = S_{m-k}$, we find that characterisitc polynomial can be written as $x^k - x^{k-1} = x^0$ or $x^k - x^{k-1} - 1 = 0$. When $k=2$, we have $x^2 - x - 1 = 0$ so that $x_1 = \phi$. However, when $k > 2$, we have $x_2 < \phi$ so that $\log_{x_1}(n) < \log_{x_2}(n)$. Since we know that the maximum degree when $k=2$ is less than the maximum degree when $k>2$, we see that delete-min has worse constants in the case when $k>2$. 
\end{sol}

\section{Problem 3}

On tradeoffs in the heap operations.

\begin{prob}
Let $P$ be a priority queue that performs insert, delete-min, and merge in $O(\log n)$ time, and performs make-heap in $O(n)$ time where $n$ is the size of the resulting  priority queue. Show that $P$ can be modified to perform insert in $O(1)$ amortized time, without affecting the cost of delete-min or merge (i.e. $O(\log n)$ amortized time). Assume that the priority queue does not support an efficient decrease-key operation.
\end{prob}

\begin{sol}
We will store a linked list $L$ of priority queues. The original priority queue $P$ will be at the front of the linked list. When performing an insert, we will create a new priority queue and append it to $L$. When merging a new priority queue $P_{new}$, we first consolidate all of the priority queues which are not $P$, hence all the priority queues which are single nodes and have been created by an insert, and create an auxiliary priority queue $P_a$. Next, we merge $P_a$ with $P$ so that the entire data structure is now composed of a single heap $P$. Then, we merge the new priority queue $P_{new}$ with the existing queue $P$. To perform a delete-min, we first consolidate all of the roots into a single priority queue. This can be done with the same routine used in merge, i.e. creating a priority queue of the singular roots and merging that auxiliary priority queue with $P$. Then, delete min can be performed on $P$ in the usual manner.

To analyze this data structure, we will introduce a potential function $\Phi = \text{\# of roots}$. The real cost of an insert is just a constant $O(1)$, and the change in potential is 1, so $a_i = O(1)$ for insert. To examine merge and delete-min, we will first examine the consolidation subroutine. In the consolidation subroutine, a make-heap operation is performed on all single-node roots. The real cost of the make-heap is $O(c)$, where $c$ is the number of single-node roots. The merge part of the consolidation requires $O(\log n)$ real cost. The change in potential is given by $-c$, since there are now $c$ fewer roots. Thus, the amortized cost of the consolidation subroutine is $O(\log n) + c - c = O(\log n)$. This means that the merge operation, which will merge another tree in $O(\log n)$ (and won't affect the potential), will cost $O(\log n) + O(\log n) = O(\log n)$ amortized time. The delete-min operation will perform a merge, then a regular delete-min from the priority queue $P$, which will take $O(\log n) + O(\log n) = O(\log n)$ amortized time.

Thus, we have created a data structure will allows insert in $O(1)$ but retains the amortized cost of delete-min and merge.
\end{sol}

\begin{prob}
Using the above technique, show that even binary heaps can be modified to support insert in $O(1)$ amortized time while maintaining an $O(\log n)$ time bound for delete-min. Note that binary heaps do not support merge in $O(\log n)$ time.
\end{prob}

\begin{sol}
We will start out with a linked list $L$ of priority queues as before. When we insert a node, we will simply create a new root and append it to $L$. We will also keep a priority queue $Q$ which holds the roots of all other priority queues, and at first it will only contain the root of $P$. To perform delete-min, we will first perform a make-heap on all of the single-node roots, and insert the new heap's root into $Q$. After the new heap's root has been inserted into $Q$, we perform a delete-min on $Q$, then a delete-min on the tree which was the root of $Q$.

We will use a potential function $\Phi = \text{\# of roots}$ for the analysis. Insert requires amortized time $a_i = 1 + 1$, since it requires $1$ in real cost to create a new node and the potential changes by $1$. For delete-min, we first perform a make-heap on all of the single-node roots. Let's say that there exist $c$ single-node roots, then the make heap requires $O(c)$ time. Inserting the heap into $Q$ will require $O(\log n)$ time, since $Q$ has a maximum size of $n$ and therefore a maximum depth of $O(\log n)$. Delete-min on $Q$ also requires $O(\log n)$. The next delete-min on the resulting heap requires $O(\log n)$ time it must also be of size $O(n)$, and hence height $O(\log n)$. The total amortized cost of delete-min is therefore $a_i = 3 O(\log n) + c - c = O(\log n)$. 

We have therefore achieved insert in $O(1)$ amortized time and delete-min in $O(\log n)$ time for a binary heap.
\end{sol}

\section{Problem 5}

\begin{prob}
Show how to use the techniques of persistent data structures to preprocess a tree in $O(n \log n)$ time so as to allow LCA queries to be answer in $O(\log n)$ time. Aim for a simple solution here, even if you solve part (b). 
\end{prob}

\begin{sol}
We will use a persistent version of the offline algorithm given in the problem. First, we associate with each node an extra field "name" and process the nodes of $T$ is post order. However, we will make the tree into a persistent data structure by keeping a timestamped log in a fat node and using path copying for each union find we use. 

Thus, as we perform union-find operations in the preprocessing stage, we will be using path compression and fat nodes to store previous changes in the "name" attribute. This means that find will take $O(\log t)$ to find the correct tree to use and an additional $O(\log n)$ to find the node in that particular tree. Thus, there is a total real cost of $O(\log t) + O(\log n)$ to perform a find operation. To construct the data structure, we will have a real cost of $O(\log n)$ for each find and union operation, since find takes $O(\log n)$ and dominates the union operation. This means that preprocessing costs $O(\log n)$ per node. Since there are a total of $n$ find operations in the sequence, we will have $O(n \log n)$ cost for the preprocessing. 

We will use the potential function $\Phi = \text{\# live nodes}$. Thus, the amortized time of find will be $a_i = O(\log t) + O(\log n) = O(\log n)$ because $t = O(n)$ as only one operation is done for each node in the preprocessing phase. Thus, we see that find will require $O(\log n)$ time. Next, we know that the preprocessing time will require $O(n)$ times the cost of make set, union, and find on a node. Since each of these costs $O(\log n)$ time in a union-find data structure, we see that the preprocessing time is $O(n)$.

This means that we can perform LCA queries in $O(\log n)$ and preprocess the tree in $O(n \log n)$ time.
\end{sol}

\end{document}
