\documentclass[psamsfonts]{amsart}

%-------Packages---------
\usepackage{amssymb,amsfonts}
\usepackage[all,arc]{xy}
\usepackage{enumerate}
\usepackage[margin=1in]{geometry}
\usepackage{amsthm}
\usepackage{theorem}
\usepackage{verbatim}
\usepackage{tikz}
\usetikzlibrary{shapes,arrows}

\newenvironment{sol}{{\bfseries Solution:}}{\qedsymbol}
\newenvironment{prob}{{\bfseries Problem:}}

\bibliographystyle{plain}

\voffset = -10pt
\headheight = 0pt
\topmargin = -20pt
\textheight = 690pt

%--------Meta Data: Fill in your info------
\title{6.854 \\
Advanced Algorithms \\
Problem Set 1}

\author{John Wang}

\begin{document}

\maketitle

Collaborators: Jason Hoch, Ryan Liu, Varun Ganesan

\section{Problem 1}

\begin{prob}
Unlike regular heaps, Fibonacci heaps do not achieve their good performance by keeping the depth of the heap small. Demonstrate this by exhibiting a sequence of Fibonacci heap operations on $n$ items that produce a heap-ordered tree of depth $\Omega(n)$.
\end{prob}

\begin{sol}
Consider the following recursive series of operations. For the $i$th step of the recursion, we will assume there is a tree $t_1$ of depth $i$, composed of exactly $i$ nodes. There is also a tree $t_2$ which is a single node such that $root(t_1) < root(t_2)$. We shall insert two nodes $a$ and $b$ such that $b < a < root(t_1) < root(t_2)$. Perform a delete-min operation on the set of trees. This operation will remove $b$ since it is the minimum of the entire structure. This leaves us with 3 roots, namely $a, root(t_1),$ and $root(t_2)$. The delete-min operation will also perform a consolidation, so that $a$ is merged with $t_1$, then the resulting tree is merged with $t_2$. 

The resulting tree has a root of $a$, a left child of $root(t_1)$ and a right child of $root(t_2)$. Now, we perform a decrease key on $root(t_2)$ to a value lower than $a$. This will cut it off from the tree, and we will be left with a tree $t'_1$ rooted at $a$ and $t'_2$. We see that $t'_1$ will have a depth of $i+1$ and $t'_2$ will be a single node. Thus, we are back to our original datastructure with step $i+1$ and can recurse. 

Note that we performed four operations: insert $a$, insert $b$, delete-min, decrease-key. Thus, we see that after $n$ of these operations, we will have a tree of length $n/4 = \Omega(n)$. 
\end{sol}

\section{Problem 2}

\begin{prob}
Suppose that Fibonacci heaps were modified so that a node was cut only after losing $k$ children. Show that this will improve the amortized cost of decrease key (to a better constant) at the cost of a worse cost for delete-min (by a constant factor).
\end{prob}

\begin{sol}
First we will define our potential function as $\Phi = R + 2M/(k-1)$ where $R$ is the number of roots and $M$ is the number of mark bits. Examining the amortized cost of insert $a_i$ is given by:
\begin{eqnarray}
a_i = c + 1 + \Delta \Phi
\end{eqnarray}

Where $c$ is the number of nodes cut on a given insert (due to cascading). The real cost is $c+1$ because $c$ nodes are cut during cascading, each requiring constant time, and $+1$ because the node must be inserted into the data structure as well. The change is potential is given by:
\begin{eqnarray}
\Delta \Phi = c + \frac{2(1 -(k-1)(c-1))}{k-1}
\end{eqnarray}

Because $c$ nodes are cut during the cascading, an additional $c$ roots are created which accounts for the first $c$ term in $\Delta \Phi$. Moreover, the number of mark bits decreases by $(k-1)(c-1)$ since we cut away $c$ nodes, which means that $c-1$ of these nodes had $k-1$ mark bits already stored which were cleared when everything was cascaded. However, we added $1$ mark bit to the last node in the cascading chain, which is why we have a change of $1 - (k-1)(c-1)$ mark bits. Putting this into our expression, we obtain:
\begin{eqnarray}
a_i &=& 1 + c + c + \frac{2(1 -(k-1)(c-1))}{k-1}\\
&=& 1 + 2 + \frac{2}{k-1}\\
&=& 3 + \frac{2}{k-1}
\end{eqnarray}

Thus, when $k=2$, the cost for insert is $5$, whereas when $k > 2$, the cost for insert is less than $5$. Thus, the change improves the amortized cost of decrease key to a better constant.

We are left to show that cutting nodes only after losing $k$ children makes delete-min more expensive. 
\end{sol}

\section{Problem 3}

On tradeoffs in the heap operations

\begin{prob}
Let $P$ be a priority queue that performs insert, delete-min, and merge in $O(\log n)$ time, and performs make-heap in $O(n)$ time where $n$ is the size of the resulting  priority queue. Show that $P$ can be modified to perform insert in $O(1)$ amortized time, without affecting the cost of delete-min or merge (i.e. $O(\log n)$ amortized time). Assume that the priority queue does not support an efficient decrease-key operation.
\end{prob}

\begin{sol}

\end{sol}



\begin{sol}

\end{sol}

\end{document}
