\documentclass[psamsfonts]{amsart}

%-------Packages---------
\usepackage{amssymb,amsfonts}
\usepackage{enumerate}
\usepackage[margin=1.5in]{geometry}
\usepackage{amsthm}
\usepackage{theorem}
\usepackage{hyperref}
\usepackage{verbatim}
\usepackage{tikz}
\usetikzlibrary{shapes,arrows}

\newenvironment{sol}{{\bfseries Solution:}}{\qedsymbol}
\newenvironment{prob}{{\bfseries Problem:}}

\bibliographystyle{plain}

\voffset = -10pt
\headheight = 0pt
\topmargin = -20pt
\textheight = 690pt

%--------Meta Data: Fill in your info------
\title{Website Changes and User Behavior \\
Using Panjiva Data to Examine Code Changes}
\date{\today}
\author{John Wang \\
14.27 Final Paper}


\begin{document}

\begin{abstract}
\end{abstract}

\maketitle

\tableofcontents

\newpage

\section{Dataset Explanation}

The proprietary Panjiva Dataset comes from the back-end databases collected by Panjiva, Inc. Panjiva's website \url{http://www.panjiva.com} acts as a medium for buyers and suppliers of manufactured goods. The site provides a communication platform so that bulk buyers of a particular good can search and obtain unbiased information on factories and suppliers of that good. These two parties can then communicate and send messages over Panjiva's interface, attempting to strike a deal. 

Panjiva's competitive advantage rests in its ability to parse government import and export data in order to obtain unbiased information about suppliers. Panjiva determines a supplier reliability score and also provides recent history of a supplier's shipments, and allows buyers to search and aggregate this information easily. Most firms that use Panjiva are large to medium size buyers of components. For example, a department store would use Panjiva to search for suppliers of shirts or clothing, or a home improvement store would search for suppliers of socket wrenches. In addition, Panjiva provides data on trends in global manufacturing and shipping by leveraging the government data it already mines for individual supplier information.

The dataset used in this paper comes from the event and activity logs of Panjiva's website. Each time a user performs some significant event or activity on the Panjiva website, an entry will be created in either the event or activity log. All nontrivial features of the website, including supplier search, U.S. import and export search, and profiles views, are accounted for. 

The data in the event and activity logs are organized so that one can trace the exact user or subscribed account for which the entry. In particular, the logs contain information on the ip address of the user, the time the activity was performed, the webpage the activity occurred on, and extra data depending on the type of activity performed. 

The enormous quantity and granularity of this data enable the analysis of user-level interactions. The event logs contain about 124 million entries while the activity logs contain about 13 million entries. Moreover, the data in each of these logs extends for multiple years and represent a phase of growth for the Panjiva website. 

\end{document}
