\documentclass[psamsfonts]{amsart}

%-------Packages---------
\usepackage{amssymb,amsfonts}
\usepackage[all,arc]{xy}
\usepackage{enumerate}
\usepackage[margin=1in]{geometry}
\usepackage{amsthm}
\usepackage{theorem}
\usepackage{verbatim}
\usepackage{tikz}
\usetikzlibrary{shapes,arrows}

\newenvironment{sol}{{\bfseries Solution:}}{\qedsymbol}
\newenvironment{prob}{{\bfseries Problem:}}

\bibliographystyle{plain}

\voffset = -10pt
\headheight = 0pt
\topmargin = -20pt
\textheight = 690pt

%--------Meta Data: Fill in your info------
\title{14.27 \\
Economics of E-Commerce \\
Problem Set 1}

\author{John Wang}

\begin{document}

\maketitle

\section{Problem 1}

\begin{prob}
Suppose that demand in an industry is given by:
\begin{eqnarray}
\log(Q) = a + b \log(P)
\end{eqnarray}
with $b < 0$.  What is the price elasticity of demand? What markup (Lerner index) would a firm with a positive marginal cost choose?  What happens if the firm is selling information or some other product with zero marginal cost? Comment on the practical importance of this conclusion.
\end{prob}
\begin{sol}
First, we know that price elasticity is given by $\frac{dQ/Q}{dP/P}$ which becomes $\frac{d \log Q}{d \log P} = b$. Therefore, price elasticity of demand is $b$. Since we know that a firm with a positive marginal cost will have a lerner index equal to $-Q P'(Q)/P(Q) = 1/\epsilon$, we know that a firm would have a markup of $1/b$. If the firm is selling a product with zero marginal cost, then they will maximize:
\begin{eqnarray}
\max_{Q} P(Q)*Q 
\end{eqnarray}
Since $P(Q) = e^{-a/b} Q^{1/b}$, we see that $P'(Q) = \frac{e^{-a/b}Q^{\frac{1}{b} - 1}}{b}$, so that first order conditions give:
\begin{eqnarray}
\frac{e^{-a/b}Q^{\frac{1}{b} - 1}}{b} Q + e^{-a/b} Q^{\frac{1}{b}} &=& 0 \\
e^{-a/b}Q^{\frac{1}{b}} ( \frac{1}{b} + 1) = 0
\end{eqnarray}
Since the firm has no marginal cost, the firm can extract as much profit as 
\end{sol}

\end{document}
