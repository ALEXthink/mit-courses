\documentclass[psamsfonts]{amsart}

%-------Packages---------
\usepackage{amssymb,amsfonts}
\usepackage[all,arc]{xy}
\usepackage{enumerate}
\usepackage{mathrsfs}
\usepackage[margin=1in]{geometry}
\usepackage{thmtools}
\usepackage{verbatim}
\usepackage{multirow}


%--------Theorem Environments--------
%theoremstyle{plain} --- default
\newtheorem{prob}{Problem}[section]
\newtheorem{thm}{Theorem}[section]
\newtheorem{cor}[thm]{Corollary}
\newtheorem{prop}[thm]{Proposition}
\newtheorem{lem}[thm]{Lemma}
\newtheorem{conj}[thm]{Conjecture}
\newtheorem{quest}[thm]{Question}

\newenvironment{sol}{{\bfseries Solution}}{\qedsymbol}

\def\legendre(#1,#2){%
{#1 \overwithdelims () #2} }

\theoremstyle{definition}
\newtheorem{defn}[thm]{Definition}
\newtheorem{defns}[thm]{Definitions}
\newtheorem{con}[thm]{Construction}
\newtheorem{exmp}[thm]{Example}
\newtheorem{exmps}[thm]{Examples}
\newtheorem{notn}[thm]{Notation}
\newtheorem{notns}[thm]{Notations}
\newtheorem{addm}[thm]{Addendum}
\newtheorem{exer}[thm]{Exercise}

\theoremstyle{remark}
\newtheorem{rem}[thm]{Remark}
\newtheorem{rems}[thm]{Remarks}
\newtheorem{warn}[thm]{Warning}
\newtheorem{sch}[thm]{Scholium}


\makeatletter
\let\c@equation\c@thm
\makeatother
\numberwithin{equation}{section}

\bibliographystyle{plain}

\voffset = -10pt
\headheight = 0pt
\topmargin = -20pt
\textheight = 690pt

%--------Meta Data: Fill in your info------
\title{18.781 \\
Problem Set 5}

\author{John Wang}

\begin{document}

\maketitle

\section{Problem 1}

\begin{prob}
Solve $x^2 \equiv 21 \pmod{41}$ using Tonelli's algorithm.
\end{prob}

\begin{sol}
First, we check if $a^{(p-1)/2} = a^{20} \equiv 1 \pmod{p}$. We use repeated square to obatin the sequence $21^2 \equiv 32 \pmod{41}$, $21^4 \equiv 18 \pmod{41}$, $ 21^{8} \equiv 37 \pmod{41}$, $21^{16} \equiv 16 \pmod{41}$. Therefore, we see that $21^{20} = 21^{16} 21^4 = (18)(16) \equiv 1 \pmod{41}$. Therefore, we see that $21$ is a quadratic residue modulo 41, and we continue the algorithm.

We write $p - 1 = 40 = (8)(5) = 2^3 5$. From this, we see that $t = 5$ and $s = 3$. Now, we pick a quadratic non-residue modulo 41. We find obtain a list of quadratic residues modulo 41 and pick a number, $n = 11$, which is not a part of that list. Now, we let $c = n^t = 11^5 \equiv 3 \pmod{41}$. We see that $c^{-1} = 3$ so that $r = a^t = 21^5 \equiv 9 \pmod{41}$. In the first loop, we find that $d = r^{2^{s - i - 1}} = 9^{2^{3 - 1 - 1}} = 9^2 = 81 \equiv -1 \pmod{41}$. Therefore, we continue the algorithm and set $b = bc = 3$ and $r = r c^2 = (9)(3^2) \equiv -1 \pmod{41}$ and $c = c^2 = 9$. Now in the second loop, we find $d = (-1)^{2^{3-2-1}} = 1$ so that we stop the algorithm. We set $b = (3)(9) = 27$ and we return $a^{(t+1)/2} b$. This comes out to $21^{6/2} 27 = 21^3 27 \equiv 29 \pmod{41}$. To get the other solution, we take $-21 \equiv 12 \pmod{41}$. 

Thus, our two solutions are $x = 12, 29 \pmod{41}$.
\end{sol}

\section{Problem 2}

\begin{prob}
Let $p$ be a prime congruent to 2 modulo 3, and let $(a,p) = 1$. Show that the congruence $x^3 \equiv a \pmod{p}$ has the unique solution $x \equiv a^{(2p-1)/3} \pmod{p}$. 
\end{prob}

\begin{sol}
First we shall show existence. By Fermat's Little Theorem, we know that $a^{p-1} \equiv 1 \pmod{p}$ and $a^p \equiv a \pmod{p}$ if $(a,p) = 1$. Thus, we know that $a^{p-1} a^{p} \equiv a \pmod{p}$ by multiplying these two congruences together. This shows that $a^{2p - 1} \equiv a \pmod{p}$. Now, if $3 | 2p - 1$, then it is clear that $a^{((2p-1)/3)^3} \equiv a \pmod{p}$. However, since $p \equiv 2 \pmod{3}$, we know that $2p \equiv 1 \pmod{3}$ which is equivalent to saying that $3 | 2p-1$. This shows that $x = a^{(2p-1)/3} \pmod{p}$ is a solution to the congruence $x^3 \equiv a \pmod{p}$. 

To show uniqueness, we assume there exist two solutions $x_1$ and $x_2$ to the congruence. Then we know that $x_1^3 \equiv x_2^3 \equiv a \pmod{p}$. Thus, since $x_1^3 \equiv x_2^3 \pmod{p}$, and we know that $x_2^{-1}$ exists because $p$ is a prime, we can rewrite the expression to $(x_1 x_2^{-1})^3 \equiv 1 \pmod{p}$. Now let us set $a = x_1 x_2^{-1}$. We know that $a^3 \equiv 1 \pmod{p}$ and also that $a^{p-1} \equiv 1 \pmod{p}$ from Fermat's Little Theorem. This means that the order of $a$ must divide $3$ and $p-1$. However, since $p$ is a prime, we know that $p-1$ is even. The only number that divides $3$ and $p-1$ is thus $1$. Therefore, we know that $a^1 \equiv 1 \pmod{p}$. This implies that $x_1 x_2^{-1} \equiv 1 \pmod{p}$, which can be rewritten as $x_1 \equiv x_2 \pmod{p}$. This shows uniqueness.
\end{sol}

\section{Problem 3}

\begin{prob}
Let $f(x) = ax^2 + bx + c$ and let $D = b^2 - 4ac$ be the discriminant of this quadratic polynomial. Let $p$ be an odd prime, such that $p \nmid a$. Show that if $p | D$ then $f(x) \equiv 0 \pmod{p}$ has exactly one solution. If $p \nmid D$ the $f(x) \equiv 0 \pmod{p}$ has either 0 or 2 solutions and if $x_0$ is a solution, then $f'(x_0) \not \equiv 0 \pmod{p}$. 
\end{prob}

\begin{sol}
Divididing the congruence by $a$ yields the congruence $ x^2 + (b/a) x + c/a \equiv 0 \pmod{p}$. This can be done because $p$ is a prime. Next, we complete the square, to get the congruence $x^2 + (b/a) x + (b^2/4a^2) - (b^2/4a^2) + c/a \equiv 0 \pmod{p} $ which can be simplified to:
\begin{eqnarray}
ax^2 + bx + c &\equiv& 0 \pmod{p} \\
\left(x + \frac{b}{2a} \right)^2 &\equiv& \frac{b^2}{4a^2} - \frac{c}{a} \pmod{p} \\
\left( (2a) \left( x + \frac{b}{2a} \right) \right)^2 &\equiv& b^2 - 4ac \pmod{p}
\end{eqnarray}

Now, if $p | D$, then we know that $p | b^2 - 4ac$ by definition, which also implies that $p | ((2a)(x + (b/2a)))^2$. Since $p \nmid a$, we know that $p \nmid a^2$, and also that $p \nmid 4$ if $p > 2$. Therefore, we find that $p | (x + b/2a)^2$, so that we must have:
\begin{eqnarray}
\left( x + \frac{b}{2a} \right)^2 \equiv 0 \pmod{p}
\end{eqnarray} 

We see that $x = - b (2a)^{-1} \pmod{p}$ is the only unique solution which solves the congruence. Since $p$ is a prime, we know the inverse of $2a$ exists and is unique. Therefore, the case where $p | D$ has exactly one solution.

Now if $p \nmid D$, then we must solve $((2a)(x + b/2a))^2 \equiv b^2 - 4ac \pmod{p}$. Let us say that $x_1$ is a solution to this congruence. Then we must have $-(x_1 + b/2a) = x_2 + b/2a$ also satisfying the congruence. This implies that $x_2 = - x_1 - b/a$ will also satisfy the congruence. Thus, in the case of $p \nmid D$, we have either two solutions or zero solutions if $x_1$ is not a solution.

Now, we must show that if $x_0$ is a solution, then $f'(x_0) \not \equiv 0 \pmod{p}$. First, we rewrite the congruence $f(x) \equiv 0 \pmod{p}$ into:
\begin{eqnarray}
\left(2ax_0 + b \right)^2 \equiv b^2 - 4ac \pmod {p}
\end{eqnarray}

We know that $f'(x_0) \equiv 2a x_0 + b \pmod{p}$. Thus, if $p \nmid D$, then we know that $(2ax_0 + b)^2 \not \equiv 0 \pmod{p}$ so that clearly $f'(x) \not \equiv 0 \pmod{p}$. Now if $p | D$, we know that the square root of $(2ax_0 + b)^2 \not \equiv 0 \pmod{p}$, so that $f'(x_0) = 2ax_0 + b \not \equiv 0 \pmod{p}$. This completes the proof.
\end{sol}

\begin{prob}
Show that if $p$ is an odd prime, $e$ a natural number, and $(a,p) = 1$, then $x^2 \equiv a \pmod{p^e}$ has exactly $1 + \legendre(a,p)$ solutions.
\end{prob}

\begin{sol}
First we will show that if $(a,p) = 1$, then $x^2 \equiv a \pmod{p}$ has exactly $1 + \legendre(a,p)$ solutions. Note that there are a maximum of $2$ solutions by the previous problem. Also note thet the determinant is $D = (-4)(-a) = 4a$. First, if $p | 4a$, then we know that $\legendre(a,p) = 0$ by definition. Moreover, from the previous problem, we know that if $p | D$, there will be exactly 1 solution, so $1 + \legendre(a,p)$ correctly gives the number of solutions in this case. 

Now, if  $p \nmid 4a$, then $p \nmid D$ so there are either 0 or 2 solutions by the previous problem's results. Thus, we have $\legendre(a, p) = \pm 1$. If $\legendre(a,p) = 1$, then there is at least one solution, so there must be two solutions to $x^2 \equiv a \pmod{p}$. This means that $1 + \legendre(a,p) = 2$ gives the correct number of solutions. If $\legendre(a,p) = -1$, then there are no solutions to $x^2 \equiv a \pmod{p}$, and $1 + \legendre(a,p) = 0$ also gives the correct number of solutions. 

Now, since we have shown that $1 + \legendre(a,p)$ is the number of solutions to $x^2 \equiv a \pmod{p}$, we can use Hansel's lemma to lift the solutions to $x^2 \equiv a \pmod{p^e}$. Since we know that $f'(x_0) \not \equiv 0 \pmod{p}$ for each solution $x_0$ to $f(x_0) \equiv 0 \pmod{p}$, the conditions for Hansel's lemma are satisfied. Since Hansel's lemma shows that there is a unique $t \pmod{p}$ such that $f( a + t p^j ) \equiv 0 \pmod{p^{j+1}}$, we know there are exactly $1 + \legendre(a,p)$ solutions to the equation $f(x) \equiv 0 \pmod{p^e}$. This completes the proof.
\end{sol}

\section{Problem 4}

Which of the following congruences have solutions, and how many? To complete this problem, we use the following two lemmas proven in class:

\begin{lem}
Suppose that $m = 1, 2, 4, p^\alpha, 2p^\alpha$ where $p$ is an odd prime. If $(a,m) = 1$ then the congreuence $x^n \equiv a \pmod{m}$ has $(n, \phi(m))$ solutions or no solutions according as $a^{\phi(m)/(n, \phi(m))}, 1 \pmod{m}$. 
\label{a}
\end{lem}

\begin{lem}
Let $f(x)$ be a fixed polynomial with integral coefficients and for any positive integer $m$, let $N(m)$ denote the number of solutions of the congruence$f(x) \equiv 0 \pmod{m}$. If $m = m_1 m_2$ iwhere $(m_1, m_2) = 1$, then $N(m) = N(m_1) N(m_2)$. If $m = \prod p^\alpha$ is the canonical factorization of $m$, then $N(m) = \prod N(p^{\alpha})$. 
\label{b}
\end{lem}

These two lemmas will allow us to the determine the number of solutions the following congruences:

\begin{prob}
$x^2 \equiv - 2 \pmod{118}$
\end{prob}

\begin{sol}
First, we note that $118 = 2 \cdot 59$, where both $2$ and $59$ are primes. Because we know this, we can now apply lemma \ref{a} to determine the number of solutions to $x^2 \equiv -2 \pmod{59}$ and $x^2 \equiv -2 \pmod{2}$. We know that $\phi(59) = 58$ because it is a prime. Moreover, $\phi(59)/(2, \phi(59)) = 29$ and we see that $(-2)^{29} \equiv 1 \pmod{58}$. This means there are 2 solutions to the congruence $x^2 \equiv -2 \pmod{59}$. Next, we see from inspection that $x = 0$ is the only solution to $x^2 \equiv -2 \pmod{2}$. The total number of solution of $x^2 \equiv -2 \pmod{118}$ is therefore $2 \cdot 1 = 2$. 
\end{sol}

\begin{prob}
$x^2 \equiv -1 \pmod{244}$. 
\end{prob}

\begin{sol}
We note that $244 = 4 \cdot 61$. We now want to determine the number of solutions to $x^2 \equiv -1 \pmod{4}$ and $x^2 \equiv -1 \pmod{61}$. However, we see that $x^2 \equiv -1 \pmod{4}$ has no solutions because $x = 0,1,2,3$ do not work. This means there are no solutions to $x^2 \equiv -1 \pmod{4}$, and thus that there are no solutions to $x^2 \equiv -1 \pmod{244}$. 
\end{sol}

\begin{prob}
$x^2 \equiv -1 \pmod{365}$. 
\end{prob}

\begin{sol}
We see that $365 = 5 \cdot 73$. Thus, we need to determine the number of solutions of $x^2 \equiv -1 \pmod{5}$ and $x^2 \equiv -1 \pmod{73}$. First, we note that the number of solutions to $x^2 \equiv -1 \pmod{5}$ is given by $1 + \legendre(4,5)$ due to a lemma from class, which can be rewritten $1 + \legendre(2^2, 5) = 2$. Next, we note that $\legendre(-1, 73) = 1$ because $73 \equiv 1 \pmod{4}$ so that $x^2 = -1 \pmod{73}$ has $1 + \legendre(-1,73) = 2$ solutions. Thus, we see that $x^2 \equiv -1 \pmod{365}$ has $2 \cdot 2 = 4$ solutions.
\end{sol}

\begin{prob}
$x^2 \equiv 7 \pmod{227}$. 
\end{prob}

\begin{sol}
We note that $227$ is a prime number, so the number of solutions of the congruence is $1 + \legendre(7, 227)$. This means, we simply need to evaluate $\legendre(7,227) = \legendre(227, 7)$ using quadratic reciprocity because $227 \equiv 3 \pmod{4}$. and $7 \equiv 3 \pmod{4}$. This simplifies to $\legendre(227,7) = \legendre(3, 7) = \legendre(7,3)$, again by quadratic reciprocity. Since $\legendre(7,3) = \legendre(1,3) = 1$, we have shown that $x^2 \equiv 7 \pmod{227}$ has 2 solutions.
\end{sol}

\begin{prob}
$x^2 \equiv 267 \pmod{789}$. 
\end{prob}

\begin{sol}
We see that the prime factorization of $789$ is $789 = 3 \cdot 263$. Thus, we need to check the number of solutions to $x^2 \equiv 267 \pmod{3}$ and $x^2 \equiv 267 \pmod{263}$. These congruences simplify to $x^2 \equiv 0 \pmod{3}$ and $x^2 \equiv 14 \pmod{263}$ respectively. Clearly, $x^2 \equiv 0 \pmod{3}$ has 2 solutions because $\legendre(0, 3 ) = 1$. However, we know that $\phi(263)/(2, \phi(263)) = 262/2 = 131$ and that $14^{131} \equiv -1 \pmod{263}$. This means that $x^2 \equiv 14 \pmod{263}$ has no solutions. This implies that $x^2 \equiv 267 \pmod{789}$ also has no solutions.
\end{sol}

\section{Problem 5}

\begin{prob}
Show that for all primes $p$, the congruence $x^8 \equiv 16 \pmod{p}$ has a solution.
\end{prob}

\begin{sol}
First, we note that if $p = 2$, we can choose $x \equiv 0 \pmod{2}$, which will solve the congruence. Thus, we can prove the statement for odd primes $p$. We know from a theorem in class that if $p$ is a prime and $(a,p) = 1$, then $x^n \equiv a \pmod{p}$ has $(n, p- 1)$ solutions if and only if $a^{(p-1)/(n, p - 1)} \equiv 1 \pmod{p}$. First, we know that $(16, p) = 1$ for all primes $p$. Thus, we only have to evaluate $16^{(p-1)/(n,p-1)} \pmod{p}$. We see that $(8, p - 1)$ can only take on values $2, 4, 8$ since $p$ is an odd prime, so $p-1$ is even. 

If $(8, p-1) = 2$, then we see that $16^{(p-1)/2} = 8^{p-1} \equiv 1 \pmod{p}$ by Fermat's Little Theorem. If $(8, p-1) = 4$, then $16^{(p-1)/4} = 2^{(p-1)} \equiv 1 \pmod{p}$ also by Fermat. If $(8, p-1) = 8$, then $16^{(p-1)/8} = 2^{(p-1)/2}$. Using the definition of quadratic residues, we see that $2^{(p-1)/2} = \legendre(2,p)$. Since $(8, p-1) = 8$ implies that $p-1 \equiv 0 \pmod{8}$, we see that $\legendre(2,p) = 1$ by a lemma proven in class. This completes the proof. 
\end{sol}

\section{Problem 6}

\begin{prob}
Prove that there are infinitely many primes of the form $8k + 7$. 
\end{prob}

\begin{sol}
Suppose the contrary. Then there exists a finite set of primes $p_1, p_2, \ldots, p_m$ which are the only primes satisfying $p \equiv 7 \pmod{8}$. Now let us construct the number $N = (4 p_1 p_2 \ldots p_m)^2 - 2$. Note that $N$ is even so it must have a prime factor since it is greater than 2. Let us find an odd prime factor $q$. Then by construction, we know that $2$ is a quadratic residue modulo $q$. However, we see that $\legendre(2,q) = 1$ implies that $q = \pm 1 \pmod{8}$ by a lemma shown in class. If we assume that none of the prime factors of $N$ have the form $8k + 7$, then they must all be of the form $8k + 1$. However, this implies that $N = (8k+1)(8k+1) = 64k^2 + 16k + 1 = 8(8k^2 + 2k) + 1 = 8k' + 1$, which is a contradiction since $N = 8k + 6$. This means that some prime factor of $N$ is congruent to $7 \pmod{8}$. However, this means that $q$ is not in the list $p_1, p_2, \ldots, p_m$ because $q | N$ but $p_i \nmid N$. This is a contradiction, and thus, there must be infinitely many primes of the form $8k + 7$. 
\end{sol}

\section{Problem 7}

\begin{prob}
Determine by congruence conditions the set of primes $p$ such that $\legendre(10,p) = 1$. 
\end{prob}

\begin{sol}
First, the properties of the Jacobi symbol allow us to factor $\legendre(10,p) = \legendre(5,p) \legendre(2,p)$. By a lemma from class, we know the following:
\begin{equation}
\legendre(2,p) = \left\{ \begin{array}{c l}
-1 & \textrm{if } p \equiv \pm 3 \pmod{8} \\
1 & \textrm{if } p \equiv \pm 1 \pmod{8}
\end{array} \right.
\end{equation}

In order to evaluate $\legendre(5,p)$, we note that $5 \equiv 1 \pmod{4}$ which means we can use quadratic reciprocity to find that $\legendre(5,p) = \legendre(p,5)$. We know that half of the reduced residue system modulo 5 will be quadratic residues. First, we know that $\legendre(-1, 5) = 1$ by a lemma shown in class and trivially that $\legendre(1,5) = 1$. We know that $\legendre(2,5) = -1$ because $5 \equiv -3 \pmod{8}$. Finally, we know that $\legendre(3, 5) = \legendre(5,3) = \legendre(2,3) =-1$. Therefore, we find:
\begin{equation}
\legendre(p,5) = \left\{ \begin{array}{c l}
-1 & \textrm{if } p \equiv \pm 2 \pmod{5} \\
1 & \textrm{if } p \equiv \pm 1 \pmod{5} 
\end{array} \right.
\end{equation} 

Using the Chinese Remainder Theorem, we can then solve for all $p$ such that $\legendre(5,p) \legendre(2, p) = 1$. This occurs if $p \equiv \pm 1 \pmod{8}, p \equiv \pm 1 \pmod{5}$ or when $p \equiv \pm 3 \pmod{8}, p \equiv \pm 2 \pmod{5}$. The Chinese Remainder Theorem says there is a unique solution to each of the systems of congruences modulo $40$. The system $p \equiv 1 \pmod{8}, p \equiv 1 \pmod{5}$ has a solution $p \equiv -9 \pmod{40}$. The system $p \equiv 1 \pmod{8}, p \equiv -1 \pmod{5}$ gives the solution $-1 \pmod{40}$. Continuing in this manner, we find the following characterization of all primes $p$ such that $\legendre(10,p) = 1$:
\begin{equation}
p = \pm 1, \pm 3, \pm 9, \pm 13 \pmod{40}
\end{equation}
\end{sol}

\section{Problem 8}

\begin{prob}
Determine, by congruence conditions, the set of primes $p$ such that $-3$ is a quadratic residue mod $p$. 
\end{prob}

\begin{sol}
We need to determine the primes $p$ such that $\legendre(-3, p) = 1$. We use the properties of Legendre symbols to symplify this expression:
\begin{eqnarray}
\legendre(-3, p) &=& \legendre(-1,p) \legendre(3, p) \\
&=& \legendre(-1, p) \legendre(p, 3)  c
\end{eqnarray}

Where $c$ is a constant depending on $p$. From the lemma proven in class, we know the following:
\begin{equation}
\legendre(-1, p) = \left\{ \begin{array}{c l}
-1 & \textrm{if } p \equiv 3 \pmod{4} \\
1 & \textrm{if } p \equiv 1 \pmod{4}
\end{array} \right.
\end{equation}

From quadratic reciprocity, we know that $c$ is given by:
\begin{equation}
c = \left\{ \begin{array}{c l}
-1 & \textrm{if } p \equiv 3 \pmod{4} \\
1 & \textrm{if } p \equiv 1 \pmod{4} 
\end{array} \right.
\end{equation}

Thus, if $p \equiv 3 \pmod{4}$, then $\legendre(-1,p) = -1$ and $c = -1$, so that $\legendre(-1, p) c = 1$. If $p \equiv 1 \pmod{4}$, we see that $\legendre(-1, p) c = (1) (1) = 1$. This, we see that $\legendre(-1,p) c = 1$ in all cases. Therefore, we can further simplify out expression to:
\begin{eqnarray}
\legendre(-3,p) = \legendre(p, 3)
\end{eqnarray}

Since $\legendre(p,3)$ simplifies to $\legendre(k, 3)$ where $k \equiv p \pmod{3}$, we see that we can break expression down to evaluate $\legendre(0, 3), \legendre(1, 3),$ and $\legendre(2,3)$. Since $\legendre(0,3) = 0$, we ignore it. We see that $\legendre(1,3) = 1$ and $\legendre(2,3) = \legendre(-1, 3) = -1$ by the lemma proven in class. Therefore, we have:
\begin{equation}
\legendre(-3, p) = \left\{ \begin{array}{c l}
-1 & \textrm{if } p \equiv -1 \pmod{3} \\
1 & \textrm{if } p \equiv 1 \pmod{3} 
\end{array} \right.
\end{equation}
\end{sol}

\begin{prob}
Prove that there are infinitely many primes of the form of each of the forms $3k+1$ and $3k - 1$. 
\end{prob}

\begin{sol}
Suppose by contradiction, that there exists a finite set $S = \{k_1, k_2, \ldots, k_n\}$ for which $3k_i+1$ and $3k_i-1$ are both primes for any $i \in \{1,2,\ldots, n\}$. Let $k_n$ be the largest $k_n$ possible where $3k_n + 1$ and $3k_n - 1$ are both prime. Let $P = \{p_1, p_2, \ldots, p_m \}$ be the set consisting of all primes of the form $3k+1$ less than or equal to $3k_n + 1$ and $Q = \{q_1, q_2, \ldots, q_r\}$ be the set consist of all primes of the form $3k -1$ less than or equal to $3k_n + 1$. Now, we shall construct the number $N = (q_1 q_2 \ldots q_r p_1 p_2 \ldots p_m)^2 + 3$. Since we know that $q_i \equiv -1 \pmod{3}$, we know that $q_i^2 \equiv 1 \pmod{3}$. We also know that $p_i \equiv 1 \pmod{3}$ so that $p_i^2 \equiv 1 \pmod{3}$. This means that $(q_1 q_2 \ldots q_r p_1 p_2 \ldots p_m)^2 \equiv 1 \pmod{3}$, implying that $N \equiv 1 \pmod{3}$. This also implies that $N - 2 = (q_1 q_2 \ldots q_r p_1 p_2 \ldots p_m)^2 + 1 \equiv -1 \pmod{3}$. 

However, we see that $N$ must be prime. This is because any prime $s$ which divides $N$ would have the following congruence: $(q_1 q_2 \ldots q_r p_1 p_2 \ldots p_m)^2 \equiv -3 \pmod{s}$. By the previous problem, we see that $s \equiv 1 \pmod{3}$. However, since $p_1, p_2, \ldots, p_m$ consist of all the primes of this form, none of them can possibly divide $(q_1 q_2 \ldots q_r p_1 p_2 \ldots p_m)^2 + 1$. Therefore, we see that $N$ does not have any prime factors, and is thus a prime. 

Moreover, we see that $N - 2$ must be prime. This is because there does not exist any prime $s$ for which $s | (q_1 q_2 \ldots q_r p_1 p_2 \ldots p_m)^2 + 1$, since this is equivalent to $(q_1 q_2 \ldots q_r p_1 p_2 \ldots p_m)^2 \equiv -1 \pmod{s}$. This follows because the only primes $s$ for which this congruence is solvable are primes of the form $4k + 1$. Since the union $P \cup Q$ consists of all primes less than $3k_n + 1$, $s$ cannot belong to the set $P \cup Q$ which implies that $s$ cannot be a prime. Since $N-2$ does not have any prime factors, it must be a prime. 

We have therefore found a new pair $N$ and $N-2$ which have the forms $3k + 1$ and $3k-1$ which are prime, and which do not belong to the lists $P$ and $Q$ respectively. This is a contradiction.
\end{sol}

\section{Problem 9}

\begin{prob}
Let $p$ be an odd prime, and let $(k,p) = 1$. Show that the number of solutions $(x,y)$ to $y^2 \equiv x^2 + k \pmod{p}$ is exactly $p-1$. 
\end{prob}

\begin{sol}
We know that the congruence $y^2 \equiv x^2 + k \pmod{p}$ can be rewritten as $y^2 - x^2 \equiv k \pmod{p}$. Factoring out the left hand side, we obtain $(y-x)(y+x) \equiv k \pmod{p}$. Now we can set new variables $z = x+y$ and $w = x-y$ so that we obtain the congruence $zw \equiv k \pmod{p}$. Thus, there exists a bijection between the solutions of the congruence $y^2 \equiv x^2 + k \pmod{p}$ and the congruence $zw \equiv k \pmod{p}$. 

Now, let us fix $z$ and find the number of solutions $w$ to $zw \equiv k \pmod{p}$ is simply $gcd(z,p)$ if $gcd(z,p) | k$ and $0$ otherwise. Notice that if $z \neq p$, then $gcd(z,p) = 1$ because $p$ is a prime. Moreover, it is clear that $1 | k$. However, if $z = p$, then $gcd(z,p) = p \nmid k$. This implies that when $z = p$, there are no solutions. This means that the total number of solutions is $\sum_{z=1}^{p-1} gcd(z,p) $ since we must exclude $gcd(p,p)$ from the solutions. Since $gcd(z,p) = 1$ for $x \in \{1, 2, \ldots, p-1\}$, we see that this sum is equal to $p-1$. Thus, the congruence has $p-1$ solutions.
\end{sol}

\begin{prob}
Show that $\sum_{x=1}^p \legendre(x^2 + k, p) = -1$. 
\end{prob}

\begin{sol}
We know that the number of solutions to the congruence $y^2 \equiv x^2 + k \pmod{p}$ given a fixed $x$ is $1 + \legendre(x^2 + k, p)$ by a lemma shown in class. Thus, the total number of solutions to $y^2 \equiv x^2 + k \pmod{p}$ is given by the expression:
\begin{eqnarray}
\sum_{x=1}^p 1 + \legendre(x^2 + k, p) = p + \sum_{x=1}^p \legendre(x^2 +k, p)
\end{eqnarray}

However, we know from the previous problem that the number of solutions to the congruence is $p - 1$. This implies that $\sum_{x=1}^p \legendre(x^2 + k,p) = -1$. 
\end{sol}

\begin{prob}
Now let $(ab,p) = 1$, show that the number of solutions to the congruence $ax^2 + by^2 \equiv 1 \pmod{p}$ is $p - \legendre(-ab,p)$. 
\end{prob}

\begin{sol}
We know that $ax^2 + by^2 \equiv 1 \pmod{p}$ can be rewritten as $by^2 \equiv 1 - ax^2 \pmod{p}$. Moreover, since $p$ is a prime, we know that $b^{-1}$ exists, so we can write $y^2 \equiv b^{-1} (1 - ax^2 ) \pmod{p}$. From a lemma proven in class, we know that the number of solutions to this congruence, holding $x$ fixed, is simply $1 + \legendre(b^{-1}(1-ax^2), p)$. Thus, the total number of solutions to the congruence is the sum of this expression, ranging over all possible $x$ values. This can be written:
\begin{eqnarray}
\sum_{x=1}^p 1 +  \legendre(b^{-1}(1 - ax^2), p) &=& p + \sum_{x=1}^p \legendre(b^{-1}, p) \legendre(1 - ax^2, p ) \\
&=& p + \sum_{x=1}^p \legendre(b, p) \legendre(1 - ax^2, p ) \\
&=& p + \sum_{x=1}^p \legendre(b - bax^2, p) \\
&=& p  + \sum_{x=1}^p \legendre(-ab,p) \legendre(x^2 - a^{-1}, p) \\
&=& p + (-1) \legendre(-ab, p)
\end{eqnarray}

Which is what we wanted to prove. Note that the second step from above comes about because $\legendre(b,p)^{-1} = \legendre(b,p)$. We also know that $a^{-1}$ exists because $p$ is a prime. The final step uses the result from the previous problem that $\sum_{x=1}^p \legendre(x^2 + k, p) = -1$, where in this case $k = -a^{-1}$. 
\end{sol}

\section{Problem 10}

\begin{prob}
Write a gp program to calculate the number of quadratic residues $R$ and quadratic nonresidues $N$ in the set $\{1,2,\ldots,(p-1)/2\}$ for any given odd prime $p$. Tabulate the results for the first 100 odd primes. What do you observe? 
\end{prob}

\begin{sol}
The program to tabulate the number of quadratic residues and non residues in the set $\{1, 2, \ldots, (p-1)/2 \}$ for the first 100 odd primes is given below:
\begin{verbatim}
    printQR(p)=
    {
        QR = 0;
        QNR = 0;
        for(a=1, (p-1)/2,
            if(kronecker(a,p) == 1, QR++);
            if(kronecker(a,p) == -1, QNR++);
        );
        print("Prime " p " R: " QR " N: " QNR);
    }

    {
        plist = primes(101);
        for(i=2, 101,
             printQR(plist[i])
        );

    }
\end{verbatim}

We observe that when $p \equiv 1 \pmod{4}$, the number of quadratic residues and non-residues is the same. When $p \equiv 3 \pmod{4}$, there are more quadratic residues than non-residues in the set.
\end{sol}

\end{document}