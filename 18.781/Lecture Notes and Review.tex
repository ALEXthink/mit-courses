\documentclass[psamsfonts]{amsart}

%-------Packages---------
\usepackage{amssymb,amsfonts}
\usepackage[all,arc]{xy}
\usepackage{enumerate}
\usepackage{mathrsfs}
\usepackage[margin=1in]{geometry}
\usepackage{thmtools}
\usepackage{verbatim}
\usepackage{multirow}


%--------Theorem Environments--------
%theoremstyle{plain} --- default
\newtheorem{prob}{Problem}[section]
\newtheorem{thm}{Theorem}[section]
\newtheorem{cor}[thm]{Corollary}
\newtheorem{prop}[thm]{Proposition}
\newtheorem{lem}[thm]{Lemma}
\newtheorem{conj}[thm]{Conjecture}
\newtheorem{quest}[thm]{Question}

\newenvironment{sol}{{\bfseries Solution}}{\qedsymbol}

\def\legendre(#1,#2){%
{#1 \overwithdelims () #2} }

\theoremstyle{definition}
\newtheorem{defn}[thm]{Definition}
\newtheorem{defns}[thm]{Definitions}
\newtheorem{con}[thm]{Construction}
\newtheorem{exmp}[thm]{Example}
\newtheorem{exmps}[thm]{Examples}
\newtheorem{notn}[thm]{Notation}
\newtheorem{notns}[thm]{Notations}
\newtheorem{addm}[thm]{Addendum}
\newtheorem{exer}[thm]{Exercise}

\theoremstyle{remark}
\newtheorem{rem}[thm]{Remark}
\newtheorem{rems}[thm]{Remarks}
\newtheorem{warn}[thm]{Warning}
\newtheorem{sch}[thm]{Scholium}


\makeatletter
\let\c@equation\c@thm
\makeatother
\numberwithin{equation}{section}

\bibliographystyle{plain}

\voffset = -10pt
\headheight = 0pt
\topmargin = -20pt
\textheight = 690pt

%--------Meta Data: Fill in your info------
\title{Lecture Notes and Final Review\\
\today}

\author{John Wang}

\begin{document}

\maketitle


\section{Introduction and Mathematical Tools}

\begin{thm}
Given $a,b \in \mathrm{Z}$ such that $a > 0$, $\exists q, r \in \mathrm{Z}$ such that $b = aq + r$ and $r < a$. 
\end{thm}

\begin{proof}
Examine the set $S = \{ b - ka: b - ka > 0, a > 0, a,r \in \mathrm{Z} \}$. We shall show that this set is nonempty (and it is clearly a subset of $\mathrm{N}$). We know that $b - 0 \dot a \in S$ so if $b > 0$, then the set is nonempty. If $b < 0$, then there exists a $k$ such that $b - ka > 0$, which shows the set is nonempty. Thus, we can use the well-ordering principle so that there exists some $r$ which is a minimum in the set $S$. 

Now we shall show that $r < a$. Suppose not and $r \geq a$. Then we know that $r = b - ka \geq a$. Hwoever, we know that $b - (k+1) a \geq 0$ which means there is a smaller element in $S$, contradicting the minimality of $r$. 
\end{proof}

\begin{thm}
Let $g = gcd(a,b)$. Then $\exists x_0, y_0 \in \mathrm{Z}$ such that $a x_0  + b y_0 = g$. 
\end{thm}

\begin{proof}
Let $S = \{ ax + by : x, y \in \mathrm{Z}, ax + by > 0 \}$. We know that the set is nonempty because you can choose $x,y = 1$ so that $a + b \in S$. Therefore, we can use the well ordering property to obtain a minimum $g$ of $S$. We'll show that $g | a$ and $g | b$, and if there is any other divisor $d$ of $a$ and $b$, then $d | g$. 

Assume that $g \nmid a$. Then there exists an $r > 0$ such that $a = gq + r$ where $r < g$. This means that $g = ax + by = (gq + r)x + by$. This means that $r = a ( q - xq) - byq$. This shows that $r \in S$, which is a contradiction because $r < g$ which contradicts the minimality of $g$. We see that $g | b$ follows similarly. 

Now assume that $d | a$ and $d|b$, then $d | ax + by = g$ as well by the properties of division. 
\end{proof}

\begin{con}
Euclidean Algorithm: Given two integers $a,b$:
\begin{enumerate}
\item If $a$ or $b$ is negative, replace it by its negative.
\item If $a > b$, switch $a,b$ so that $a \leq b$.
\item If $a = 0$, return $b$.
\item Since $b \geq a$, write $b = aq + r$ where $0 \leq r < a$ and replace $(a,b)$ with $(r,a)$, and loop on $3$. 
\end{enumerate}
\end{con}

\begin{thm}
Fundamental Theorem of Arithmetic: Any positive integer can be written as a product of primes uniquely. 
\end{thm}

\begin{proof}
Existence. We will use induction to show existence, and suppose that that all integers less than or equal to $n$ can be written as a product of primes. If $n+1$ is a prime, then we are done because it can be written as $(1)(n+1)$. If $n+1$ is not a prime, then it is composite and can be decomposed into integers $k,q \leq n$ such that $kq = n$. Since $k$ and $q$ can be written as a product of primes, we can write $n+1$ as a product of primes as well. This completes the induction step.

Uniqueness. Suppose there are two ways to write $n$ as a product of primes: $n = p_1 p_2 \ldots p_r = q_1 q_2 \ldots q_s$. 
\begin{lem}
If $p$ is a prime and $p | ab$, then $p|a$ or $p | b$. 
\end{lem}

\begin{proof}
Suppose $p \nmid a$, then we know that $gcd(p,a) = 1$ because $p$ is a prime. We see that $p|ab$ which implies that $p | b$. 
\end{proof}

Therefore, we see that $p_1 | q_i$ for some $i \in \{1, 2, \ldots,s \}$. However, since both $p_1$ and $q_i$ are primes, we see that $p_1 = q_i$. This means that we can cancel $p_1$ and $q_i$, and obtain $p_2 \ldots p_r = q_1 \ldots q_{i-1} q_{i+1} \ldots q_s$. Continuing downwards, we find that this happens for all the primes on the list, so that $p_1 \ldots p_r$ is just a reordering of $q_1 \ldots q_s$. 
\end{proof}

\begin{thm}
Euclids Infinitude of Primes: There exist an infinite number of primes.
\end{thm}

\begin{proof}
Suppose by contradiction that this is not true. Then we can enumerate the primes $S = \{p_1, p_2, \ldots, p_n\}$. Then we can construct the number $N = p_1 p_2 \ldots p_n + 1$. We know that $p_i \nmid N$ for all $i \in \{1, \ldots, n\}$ because $gcd(p_i, N) = 1$. This implies that no prime divides $N$. Moreover, $N$ cannot be a prime because it is not listed in the set $S$. This means that it must be composite, and hence, it can be decomposed into a product of primes. However, we see no primes in $S$ can be factors of $N$. This is a contradiction.
\end{proof}

\begin{thm}
$p^e || n!$ where $e = \lfloor \frac{n}{p} \rfloor + \lfloor \frac{n}{p^2} \rfloor + \ldots$. 
\end{thm}

\begin{proof}
Consider the set $S = \{1,2, \ldots, n\}$ which is the complete residue system modulo $n$. We know that $n! = n (n-1) (n-2) \ldots 1$. Moreover, we know that $\lfloor \frac{n}{p} \rfloor$ is the number of multiples of $p$ in the set $S$. Likewise, $\lfloor \frac{n}{p^2} \rfloor$ is the number of multiples of $p^2$ in the set $S$. Thus, we see that $e = \lfloor \frac{n}{p} \rfloor + \lfloor \frac{n}{p^2} \rfloor + \ldots$ is the total number of multiples of $p$ in the set $S$. This means that $p^e$ divides evenly into the product of $S$ so that $p^e || n!$. 
\end{proof}

\section{Congruences}

\begin{lem}
If $r_1, r_2, \ldots, r_k$ is a reduced residue system modulo $m$ and $gcd(a,m) = 1$, then so is $a r_1, a r_2, \ldots, a r_k$. 
\end{lem}

\begin{proof}
We need to show that $gcd(a r_1, m) = 1$. This is true because $gcd(a,m) =1$ by assumption and $gcd(r_i, m) = 1$ by the definition of a reduced residue system. This implies that $gcd(a r_i, m) = 1$. Now we need to show that all $a r_i$ are distinct modulo $m$. Suppose not. Then $a r_i \equiv a r_j \pmod{m}$ for some $i \neq j$. Then we see that $a (r_i - r_j) \equiv 0 \pmod{m}$. Since $gcd(a,m) = 1$, we know that $m \nmid a$ so that $m | r_i - r_j$. This implies that $r_i \equiv r_j \pmod{m}$ which is a contradiction.
\end{proof}

\begin{thm}
If $gcd(a,m) = 1$, then $a^{\phi(m)} \equiv 1 \pmod{m}$. 
\end{thm}

\begin{proof}
We will prove this by invoking the lemma from above. Let $r_1, r_2, \ldots, r_k$ be a reduced residue system modulo $m$. Then we see that $a r_1  a r_2 \ldots a r_k \equiv r_1 r_2 \ldots r_k \pmod{m}$. This shows that $a^{k} \equiv 1 \pmod{m}$. Since $k = \phi(m)$ is the number of objects in the reduced residue system, we are finished with our theorem.
\end{proof}

\begin{cor}
Fermat's Little Theorem: If $gcd(a,m) = 1$, then $a^p \equiv a \pmod{p}$. 
\end{cor}

\begin{lem}
The congruence $x^2 \equiv 1 \pmod{p}$ has only solutions $x \equiv \pm 1 \pmod{p}$.
\end{lem}

\begin{proof}
It is clear that $x^2 - 1 \equiv 0 \pmod{p}$ is another way to write the above equation. Factoring out the left side, we see that $(x-1)(x+1) \equiv 0 \pmod{p}$. This means that $p | (x-1)(x+1)$. Moreover, since $p \geq 2$, we know that $p | x-1$ or $p | x + 1$. Thus, the only solutions to the equation come about when $x \equiv \pm 1 \pmod{p}$.
\end{proof}

\begin{thm}
Wilson's Theorem. If $p$ is a prime, then $(p-1)! \equiv -1 \pmod{p}$.
\end{thm}

\begin{proof}
We know that $\{1,2,\ldots, p\}$ is a reduced residue system modulo $p$, since $\phi(p) = p-1$. Since we know that $a \equiv a^{-1} \pmod{p}$, which implies $a a^{-1} \equiv 1 \pmod{p}$ unless $a \equiv \pm 1 \pmod{p}$. This means that we can pair up elements in the system with their inverses and obtain:
\begin{eqnarray}
(a_1 a_1^{-1})(a_2 a_2^{-1}) \ldots (a_k a_k^{-1}) \equiv (-1)(1) \equiv -1 \pmod{p}
\end{eqnarray}

This follows because the only factors of $(p-1)!$ which cannot be grouped into pairs equivalent to $1 \pmod{p}$ are $1$ and $-1$. The theorem follows.
\end{proof}

\begin{thm}
The congruence $x^2 \equiv - 1 \pmod{p}$ is solvable if and only if $p = 2$ or $p \equiv 1 \pmod{4}$. 
\end{thm}

\begin{proof}
The theorem follows trivially in the case when $p = 2$. Now, we will assume that $x^2 \equiv -1 \pmod{p}$. Assume by contradiction that there is a solution if $p \equiv 3 \pmod{4}$. We know that $p-1 \equiv 2 \pmod{4}$. This implies that $p-1 = 4k + 2$ for some $k \in \mathrm{N}$. Thus, we find that $x^{p-1} = x^{2(2k+1)} = (x^2)^{2k+1}$. Since we know that $x^2 \equiv -1 \pmod{p}$, we see that $x^{p-1} \equiv (-1)^{2k+1} \equiv -1 \pmod{p}$ because $2k+1$ is odd. However, we see that $x^{p-1} \equiv 1 \pmod{p}$ by Fermat, which is a contradiction.

Now we shall assume that $p \equiv 1 \pmod{4}$. Then we know that $(p-1)! \equiv - 1 \pmod{p}$ by Wilson's theorem. Now we can write this as:
\begin{eqnarray}
(p-1)! &=& \left(1 \ddot 2 \ddot 3 \ldots \frac{p-1}{2} \right) \left( \frac{p+1}{2} \cdot \frac{p+3}{2} \ldots (p-1) \right) \equiv -1 \pmod{p}
\end{eqnarray}

Let $x = 1 \ddot 2 \ddot 3 \ldots \frac{p-1}{2}$, and so we need to show that $x \equiv \frac{p+1}{2} \cdot \frac{p+3}{2} \ldots (p-1) \pmod{p}$. Yet we know that $p - 1 \equiv (-1)(1) \pmod{p}$, $p-2 \equiv (-1)(2) \pmod{p}$, \ldots, $\frac{p+1}{2} \equiv (-1)(\frac{p-1}{2}) \pmod{p}$. This shows that $ \frac{p+1}{2} \cdot \frac{p+3}{2} \ldots (p-1) \pmod{p} \equiv (-1)^(p-1) (1 \cdot 2 \cdot 3 \ldots \frac{p-1}{2} ) \equiv x \pmod{p}$. This completes the theorem.
\end{proof}

\begin{thm}
Chinese Remainder Theorem: Given a system of congruences $x \equiv a_i \pmod{m_i}$ for $i \in \{1, \ldots, n\}$ such that all $m_i$ are coprime in pairs, there exists a unique solution modulo $m_1 m_2 \ldots m_n$.
\end{thm}

\begin{proof}
First, we will show existence by constructing a number $a$ which satisfies all of the congruences. Let $N_i = m_1 m_2 \ldots m_{i-1} m_{i+1} \ldots m_n$. We know that $gcd(N_i, m_i) = 1 $ because of the pairwise coprimeness. We can choose $H_i$ such that $H_i N_i \equiv 1 \pmod{p}$. Now, we can set $a = H_1 N_1 a_1 + H_2 N_2 a_2 + \ldots H_n N_n a_n$. It is obvious that $H_i N_i \equiv 0 \pmod{m_j}$ for all $j \neq i$, but we know that $H_i N_i \equiv 1 \pmod{m_i}$. This means that $a \equiv a_i \pmod{m_i}$ for all $i$. This completes the construction.

Second, we will show uniqueness. Suppose there are two solutions $x$ and $y$ such that $x \equiv a_i \pmod{m_i}$ and $y \equiv a \pmod{m_i}$. This shows that $x \equiv y \pmod{m_i}$, which shows that $x - y \equiv 0 \pmod{m_i}$. This implies that $m_i | x - y$ for all $i \in \{1,2, \ldots, n\}$. However, we know that $gcd(m_i, m_j) = 1$ for all $i \neq j$. This shows that $m_1 m_2 \ldots m_n | x - y$. This means that $x \equiv y \pmod{m_1 m_2 \ldots m_n}$. This shows uniqueness. 
\end{proof}

\begin{lem}
For a polynomial $f(x) \in \mathrm{Z}[x]$, we must have $f( a + t p^j) \equiv f(a) + t p^j f'(a) \pmod{p^{j+1}}$ for a prime $p$. 
\end{lem}

\begin{proof}
We can take the taylor expansion of $f(a+tp^j)$, and we obtain:
\begin{eqnarray}
f(a + t p^j) = f(a) + tp^j f'(a) + \frac{( t p^j)^2 f''(a)}{2!} + \ldots 
\end{eqnarray}

We know that $p^{j+1} | p^{kj}$ as long as $k \geq 2$. Moreover, we see that $f^{(k)}(a)/k!$ is an integer. This is because for any monomial we have $f^{(k)}(a) = (n)(n-1) \ldots (n-k) a^{n-k}$. This means that $f^{(k)}(a)/k! = {n \choose k} a^{n-k}$, which is obviously an integer.
\end{proof}

\begin{thm}
Hansel's Lemma: Suppose that we have a solution $x = a$ of the polynomial $f(x) \equiv 0 \pmod{p^j}$. Suppose that $f(x) \in \mathrm{Z}[x]$, $f(a) \equiv 0 \pmod{p^j}$, and $f'(a) \not \equiv 0 \pmod{p}$. Then there exists a unique $t \pmod{p}$ such that $f(a + t p^j) \equiv 0 \pmod{p^{j+1}}$.  
\end{thm}

\begin{proof}
Using the lemma, we know that $f(a + t p^j) \equiv f(a) + t p^j f'(a) \pmod{p^{j+1}}$. We want to set $f(a) + t p^j f'(a) \equiv 0 \pmod{p^{j+1}}$. This is equivalent to $t f'(a) + \frac{f(a)}{p^j} \equiv 0 \pmod{p}$. This means we can find a unique $t \equiv - \left( \frac{f(a)}{p^j} \right) \frac{1}{f'(a)} \pmod{p}$. This completes the proof.
\end{proof}

\section{Primitive Roots}

\begin{lem}
Let $p$ be a prime. Suppose that $q^e || p-1$ for some prime $q$. Then there exists an element modulo $p$ of order $q^e$. 
\end{lem}

\begin{proof}
Consider the solutions of $x^{q^e} \equiv 1 \pmod{p}$. We know that $q^e | p- 1$. We know that $x^{q^e} - 1$ has exactly $q^e$ roots modulo $p$. If $\alpha$ is any such root, then $ord_p(a) | q^{e}$. Thus, if $ord_p(a) \neq q^e$, then we know that $ord_p(a) | q^e - 1$. Then we must have $\alpha$ be a root of $x^{q^{e - 1}} \equiv 1 \pmod{p}$, which has exactly $q^{e - 1}$ solutions. Since $q^e - q^{e - 1} > 0$, we know there exists $\alpha$ such that $ord_p(\alpha) = q^e$. 
\end{proof}

\begin{thm}
There exist primitive roots modulo $p$ where $p$ is a prime.
\end{thm}

\begin{proof}
Write $p - 1 = q_1^{e_1} \ldots q_r^{e_r}$. The lemma says that exists $g_i$ such that $ord_p(g_i) = g_i^{e_i}$. Now let $g = g_1 g_2 \ldots g_r$. By the lemma above, $g$ has order $q_1^{e_1} \ldots q_r^{e_r} = p - 1$, because $q_1^{e_1} \ldots q_r^{e_r}$ are all coprime. Since $\phi(p) = p -1$, we see that $g$ is a primitive root modulo $p$. 
\end{proof}

\begin{thm}
There's a primitive root modulo $m$ if and only if $m = 1,2,4, p^e, 2 p^e$ where $p$ is an odd prime. 
\end{thm}

\section{Quadratic Reciprocity}

\begin{thm}
$\legendre(a,p) = a^{(p-1)/2} \pmod{p}$ if $p \nmid a$ and $p$ is odd.
\end{thm}

\begin{proof}
We know that $a^{p-1} \equiv 1 \pmod{p}$ by Fermat's Little Theorem. Since $p-1$ is even, we must have $a^{((p-1)/2)^2} \equiv 1 \pmod{p}$. This implies that $a \equiv \pm 1 \pmod{p}$. Now let $g$ be a primitive root modulo $p$. We know that $\{1,g,g^2, \ldots, g^{p-1}\}$ runs through the entire residue system. This means that $a \equiv g^k \pmod{p}$ for some $k$. We also know that $a \equiv g^{k + m(p-1)} \pmod{p}$ so that $k$ is only defined modulo $p-1$. 

Now we know that $a$ is a quadratic residue modulo $p$ if and only if $k$ is even so that $g^k \equiv (g^{k/2})^2 \pmod{p}$. Now look at $a^{(p-1)/2} \equiv g^{k (p-1)/2} \pmod{p}$. We know that $g^{k (p-1)/2} \equiv 1 \pmod{p}$ if and only if $p-1 | k (p-1)/2$. This occurs if and only if $p-1 | k$, which occurs when $2 | k$. Thus, we see that $a^{(p-1)/2} \equiv 1 \pmod{p}$ exactly when $a$ is a quadratic residue modulo $p$. 
\end{proof}


\end{document}