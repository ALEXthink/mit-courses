\documentclass[psamsfonts]{amsart}

%-------Packages---------
\usepackage{amssymb,amsfonts}
\usepackage[all,arc]{xy}
\usepackage{enumerate}
\usepackage{mathrsfs}
\usepackage[margin=1in]{geometry}
\usepackage{thmtools}
\usepackage{verbatim}
\usepackage{multirow}


%--------Theorem Environments--------
%theoremstyle{plain} --- default
\newtheorem{prob}{Problem}[section]
\newtheorem{thm}{Theorem}[section]
\newtheorem{cor}[thm]{Corollary}
\newtheorem{prop}[thm]{Proposition}
\newtheorem{lem}[thm]{Lemma}
\newtheorem{conj}[thm]{Conjecture}
\newtheorem{quest}[thm]{Question}

\newenvironment{sol}{{\bfseries Solution}}{\qedsymbol}


\theoremstyle{definition}
\newtheorem{defn}[thm]{Definition}
\newtheorem{defns}[thm]{Definitions}
\newtheorem{con}[thm]{Construction}
\newtheorem{exmp}[thm]{Example}
\newtheorem{exmps}[thm]{Examples}
\newtheorem{notn}[thm]{Notation}
\newtheorem{notns}[thm]{Notations}
\newtheorem{addm}[thm]{Addendum}
\newtheorem{exer}[thm]{Exercise}

\theoremstyle{remark}
\newtheorem{rem}[thm]{Remark}
\newtheorem{rems}[thm]{Remarks}
\newtheorem{warn}[thm]{Warning}
\newtheorem{sch}[thm]{Scholium}

\makeatletter
\let\c@equation\c@thm
\makeatother
\numberwithin{equation}{section}

\bibliographystyle{plain}

\voffset = -10pt
\headheight = 0pt
\topmargin = -20pt
\textheight = 690pt

%--------Meta Data: Fill in your info------
\title{18.781 \\
Problem Set 1}

\author{John Wang}

\begin{document}

\maketitle

\section{Problem 1}

\begin{prob}
Let $a >0$ and $b$ be integers. Show that there is an integer $k$ such that $b + ka > 0$. 
\end{prob}

\begin{sol}
Let us examine the set $S = \{ b + ka | b +ka >0; a, b \in \mathrm{Z}; a > 0\}$. This set is nonempty because if $b \geq 0$, then we can simply take $k=1$ and $b + ka > 0$. Otherwise, if $b < 0$, then there exists some $k$ such that $b + ka > 0$. This is because $b = qa + r < (q+1)a$ for some $q \in \mathrm{Z}$ and $0 \leq r < a$. Thus, simply take $k = q + 1$ and we see that $b + ka > 0$. Thus, the set $S$ is nonempty and one can use the well ordering principle to select the smallest element from the set. This shows the existence of an integer $k$ for which $b + ka > 0$. 
\end{sol}

\section{Problem 2}

\begin{prob}
Let $a$ and $b$ be positive integers whose gcd is 1. Find the largest positive integer $n(a,b)$ which is not a non-negative integer linear combination of $a$ and $b$.
\end{prob}

\begin{sol}
Let us examine the sets $S = \{ ax + by | 0 \leq x < b; y \geq 0 \}$ and $U = \{ ax + by | ax+by > 0; 0 \leq x < b; y < 0 \}$. One can see that the set $S$ contains all the integers which can be expressed as a non-negative linear combination of $a$ and $b$. Also, the set $U$ spans the positive integers which can be expressed as negative linear combinations of $a$ and $b$. It is clear that the largest positive integer $n$ which cannot be expressed as a non-negative integer linear combination of $a$ and $b$ is the maximum element in $U$.

Therefore, we must first show that $U$ is nonempty and invoke the well ordering principle to select the maximum. First, we can assume without loss of generality that $a > b$ because $(a,b) = 1$. Then, we can simply choose $y = -1$ and we see that $ax - b \in U$. Since $0 \leq x < b$, we see that $x \geq 1$ which means that $ax - b > 0$ and is therefore in $U$. Because the set is nonempty, we can select its maximum. 

In fact, the maximum is when $x = b - 1$ and $y = -1$ because both $x, y \in \mathrm{Z}$. Therefore, we see that the maximum positive integer which cannot be represented as a non-negative linear combination of $a$ and $b$ is:
\begin{eqnarray}
a(b-1) + b(-1) &=& a(b-1) - b \nonumber \\
&=& ab - a - b
\end{eqnarray} 

Therefore, we have found that $n = ab - a - b$.
\end{sol}

\section{Problem 3}
\begin{prob}
Let $a > 1$ be a positive integer and $m,n$ be natural numbers. Show that $a^m - 1 | a^n - 1$ if and only if $m | n$. 
\end{prob}

\begin{sol}
First we shall assume $m|n$ and show that $a^m - 1 | a^n - 1$. Since $m | n$, we see that $n = dm$ for some $d \in \mathrm{Z}$. Therefore, we have $a^n - 1 = a^{md} - 1$. Moreover, we can factor $a^{md} - 1$ into the following:
\begin{eqnarray}
a^{md} - 1 &=& a^{md} - 1 + (a^{m(d-1)} - a^{m(d-1)} + \ldots + a^{m} - a^{m}) \\
&=&  (a^m - 1)(a^{m(d - 1)} + a^{m(d-2)} + \ldots + a^{m} + 1)
\end{eqnarray}

But this implies that $a^m - 1$ is a factor of $a^{md} - 1$ and therefore that $a^m - 1 | a^{md} - 1 = a^{n} - 1$.  

Now we shall assume $a^{m} - 1 | a^n - 1$ and prove that $m|n$. First, we note that if $a^{m} - 1 | a^n - 1$, then $a^{m} - 1 | a^n - 1 - (a^m - 1)$. Therefore, we see that:
\begin{eqnarray}
a^m - 1 &|& a^{n} - a^m \\
a^m - 1 &|& a^m (a^{n - m} - 1) 
\end{eqnarray}

Since we assumed that $a > 1$, we see that $(a^m - 1, a^m) = 1$. Therefore, we know that $a^m - 1 \nmid a^m$ so we can write:
\begin{eqnarray}
a^m - 1 &|& a^{n - m} - 1 \\
a^m - 1 &|& a^{n - m} - 1 - (a^m - 1) \\
a^m - 1 &|& a^{n - m} - a^m \\
a^m - 1 &|& a^m (a^{n - 2m} - 1) 
\end{eqnarray}

By the same argument as above that $(a^m - 1, a^m) = 1$, we see that $a^m - 1 | a^{n - 2m} - 1$. If we iterative this argument, we see that this will eventually terminate because the exponent is strictly decreasing. Therefore, there will be some $d \in \mathrm{N}$ at which we terminate and for which $a^m - 1 | a^{n - dm} - 1$. But we see that this terminates exactly when $n - dm = m$. Therefore, we see that $n = m(d + 1)$ for some $d > 0$. Thus, we see that $m | n$. 
\end{sol}

\begin{prob}
Show that $(a^m - 1, a^n - 1) = a^{(m,n)} - 1$.
\end{prob}

\begin{sol}
We will use the same proof structure as above. We know that $(a,b) = (a, b-a)$ if $b > a$. We assume WLOG that $n > m$ so that $(a^{m} - 1, a^n - 1) = (a^m - 1, a^n - a^m)$. This means we have:
\begin{eqnarray}
(a^m - 1, a^n - 1) &=& (a^m - 1, a^m(a^{n-m} - 1)) \\
&=& (a^m - 1, a^{n - m} - 1) 
\end{eqnarray}

By the same argument as before, namely that since $a > 1$, then $(a^m - 1, a^m) = 1$ so that $a^m - 1 \nmid a^m$. Iterating this process is simply the Euclidean algorithm on the exponents, which means that we will eventually reach $a^{(m,n)} - 1$. 
\end{sol}

\section{Problem 4}

\begin{prob}
Use the Euclidean algorithm to find an integer solution $(x_0,y_0)$ to $89x + 43 y = 1$. Then use your solution to describe all possible integer solutions systematically.
\end{prob}

\begin{sol}
The Euclidean algorithm will be used in the table below:
\begin{table}[h!]
\begin{tabular}{c | c | c  c}
Quotient & Divisor & \multicolumn{2}{c}{Vector} \\
\hline
 & 89 & 1 & 0 \\
2 & 43 & 0 & 1 \\
14 & 3 & 1 & -2 \\
& 1 & -14 & 29
\end{tabular}
\caption{Euclidean Algorithm for $89 x + 43 y = 1$}
\end{table}

Thus, we can use $x_0 = -14$ and $y_0 = 29$ to obtain $(89)(-14) + (43)(29) = 1$. Moreover, every integer solution can be described by subtracting $x_0$ and $y_0$ from $89$ and $43$ respectively. This shows that $x = 43 a + 29$ and $y = -89 a - 60$ will describe all integer solutions to $89x + 43 y = 1$. This is because when we substitute $x$ and $y$ into the equation, we obtain $89(43a + 29) + 43 (-89a - 60) = (89)(29) - (43)(60) = 1$ which is what we wanted.
\end{sol}

\section{Problem 5}

\begin{prob}
Let $1<a<b$ be integers. Show that the number of divisions steps involved in the Euclidean algorithm to compute the gcd of $a$ and $b$ is at most a universal constant times $\log(a)$. 
\end{prob}

\begin{sol}
Let us start the algorithm with $a = a_0$ and $b = b_0$. The next step of the algorithm will use the integers $a_1 < b_1$ and so on until termination at the $n+1$st step. We shall show that $a_{i+1} \leq p a_i$ for some constant $p < 1$ for all $i \in \{1, \ldots, n \}$. To show this, we choose $p = \frac{a - 1}{a}$. On the first step, since $a = a_0$, we see that $a_1 \leq p a_0 = (a-1)$. This is because $a$ is strictly decreasing in the Euclidean algorithm because $r < a$. Moreover, $a_{i+1} \leq p a_i$ for all $i \in \{1, \ldots, n \}$ because $a_i$ is strictly decreasing. This means the largest value of $a_i$ for any $i$ is $a$, and the smallest decrement occurs from $a$ to $a - 1$. Since this can only possibly occur on the first step, and since $a_{i+1} / a_i \leq (a-1)/a = p$ for all greater $i$, we see that $a_{i+1} \leq p a_i$. 

We know the algorithm terminates, so let us say that the value of the smaller number in the second to last step is $d$. Since we have shown that $a_{i+1} \leq p a_i$, we know that $d \leq p^n a$. Taking logarithms of both sides, and noting that $p < 1$, we see that $\log_{p}(d/a) \geq n$. Since $d < a$ by the strictly decreasing nature of the algorithm, we see that $n \leq \log_p (d /a) < \log_p(a) = \log(a) / \log(p) = c \log(a)$. Therefore, we have found that $n < c \log(a)$.  
\end{sol}

\section{Problem 6}

\begin{prob}
Using the math software gp/PARI, tabulate the number of primes less than $x$ for $x = 10000, 20000, \ldots, 100000$. Also tabulate the number of primes less than $x$ and of the form $4k + 1$ and the number of the form $4k +3$ and also $x / \log(x)$. 
\end{prob}

\begin{sol}
The gp/PARI code for this exercise is given below:
\begin{verbatim}
    for(i=1, 10,
        p = 0;
        p4k3 = 0;
        p4k1 = 0;
        forprime(x=1, i*10000,
            p ++;
            if((x%4) == 3, p4k3 ++, p4k1 ++);
        );
        xlogx = round(i*10000/log(i*10000));
        print("x=", i*10000);
        print("Primes: ", p);
        print("4k+1 Primes: ", p4k1);
        print("4k+3 Primes: ", p4k3);
        print("x/log(x): ", xlogx);
        print(" ");
    )
\end{verbatim}

The output shows that the number of primes of  the form $4k+1$ and $4k+3$ seem to generally be very close together. For $x = 10000$, the $4k+1$ primes have a count of $610$ while the $4k+3$ primes have $619$. This trend continues for all $x$ that were tested. Moreover, the total number of primes is equal to the sum of the primes of the form $4k+1$ and the form $4k+3$. Moreover, $x/\log(x)$ comes close to the total number of primes but gets further off as $x$ grows larger.
\end{sol}

\section{Problem 7}

\begin{prob}
A board has squares numbered 1 through $n$. Two players $A$ and $B$ play the following game:
A starts, putting a token on some square $a_1$. Then $B$ puts a token on some square $b_1$, which
is not allowed to divide $a_1$. Then $A$ follows with $a_2$, such that $a_2 \nmid a_1$ and $a_2  \nmid b_1$, and so on
(at any stage, the number of the square selected must not divide any of the previous ones).
The last person to put down a token wins. Try playing this game for $n = 10, 12, 24$. Who
wins? Prove your observation for general $n$.
\end{prob}

\begin{sol}
We will show that the first player will always win. It is clear that for $n$ small, this holds. It holds trivially for $n=1, 2, 3,4$. Now let us prove this fact for general $n$. First, we note that this game is finite and must terminate. This means that either player $A$ or $B$ must win. Let us assume by contradiction that player $B$ wins. Then he holds some winning strategy on the numbers from $\{1, 2, \ldots, n \} \setminus \{k \text{ and k's divisiors} \}$ for any $k \in \{1, \ldots, n\}$. But this means that player $A$ can start out by cancelling 1, and when player $B$ moves, he will leave a board of the form $\{1, 2, \ldots, n \} \setminus \{k \text{ and k's divisiors} \}$. We already know there exists a winning strategy on this board, since player $B$ had one. This means that Player $A$ will win, which is a contradiction. This proves that Player $A$ will always win for any $n$.
\end{sol}

\section{Problem 8}

\begin{lem}
If $N$ is of the form $4k+3$ for $k \geq 1$, then one of its prime factors must also have the form $4k + 3$.  
\end{lem}

\begin{proof}
We shall proceed by induction. For $k=1$, we find that $4k+1 = 7$ is a prime, so clearly it has a prime factor of the form $4k + 3$. Using this as a base case, let us assume that we have shown the assumption true for $k = 1, \ldots, n - 1$. We must now show that $N = 4n + 3$ has a prime factor of the form $4k +3$. 

First, if $N$ is prime, then the proof is complete. Otherwise, we can factor $N$ into components. Since $N = 4n + 3$, we can have any of the following factorizations:
\begin{eqnarray}
N = (4a + 1)(4b + 1) &=& 4(4ab + 1a + 1b) + 1\\
N = (4a + 1)(4b + 3) &=& 4(4ab + 3a + 1b) + 3\\
N = (4a + 3)(4b + 3) &=& 4(4ab + 3a + 3b + 2) + 1 
\end{eqnarray}

This follows firstly because $N = 4n + 3$ is odd, so its two factors must also be odd. Moreover, the only odd numbers possible are of the form $4k + 1$ or $4k + 3$. Now, we see that cases 1 and 3 are impossible because they are both of the form $4k + 1$. Thus, the second case is the only possibility for factoring $N$. In this case, we see that $q = (4b + 3)$ is a factor of $N$. By the inductive hypothesis, $q < N$ so that $q$ has a prime factor of the form $4k + 3$. Thus, we have shown that $N$ has a prime factor of the form $4k + 3$. 
\end{proof}

\begin{prob}
Show that there are infinitely many primes of the form $4k+3$. 
\end{prob}

\begin{sol}
Suppose there are a finite number of primes of the form $4k + 3$, denoted by $p_1, \ldots, p_n$. Then we can construct a number $d = 4 (p_1 \ldots p_n) - 1 = 4(p_1 \ldots p_n - 1) + 3$ which cannot be a prime because it is larger than any $p_i$ and is of the form $4k + 3$. From our lemma, we see that $d$ must have a prime factor of the form $4k + 3$. However, this prime factor cannot be one of $p_1, \ldots, p_n$ because $d = q p_i - 1$ for some $q \in \mathrm{Z}$. Therefore, $p_i$ does not divide evenly into $d$ for all $i \in \{1, \ldots, n\}$. Thus, it must be some other prime of the form $4k + 3$, which is a contradiction.
\end{sol}

\end{document}