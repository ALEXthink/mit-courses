\documentclass[psamsfonts]{amsart}

%-------Packages---------
\usepackage{amssymb,amsfonts}
\usepackage[all,arc]{xy}
\usepackage{enumerate}
\usepackage{mathrsfs}
\usepackage[margin=1in]{geometry}
\usepackage{thmtools}
\usepackage{verbatim}
\usepackage{multirow}


%--------Theorem Environments--------
%theoremstyle{plain} --- default
\newtheorem{prob}{Problem}[section]
\newtheorem{thm}{Theorem}[section]
\newtheorem{cor}[thm]{Corollary}
\newtheorem{prop}[thm]{Proposition}
\newtheorem{lem}[thm]{Lemma}
\newtheorem{conj}[thm]{Conjecture}
\newtheorem{quest}[thm]{Question}

\newenvironment{sol}{{\bfseries Solution}}{\qedsymbol}


\theoremstyle{definition}
\newtheorem{defn}[thm]{Definition}
\newtheorem{defns}[thm]{Definitions}
\newtheorem{con}[thm]{Construction}
\newtheorem{exmp}[thm]{Example}
\newtheorem{exmps}[thm]{Examples}
\newtheorem{notn}[thm]{Notation}
\newtheorem{notns}[thm]{Notations}
\newtheorem{addm}[thm]{Addendum}
\newtheorem{exer}[thm]{Exercise}

\theoremstyle{remark}
\newtheorem{rem}[thm]{Remark}
\newtheorem{rems}[thm]{Remarks}
\newtheorem{warn}[thm]{Warning}
\newtheorem{sch}[thm]{Scholium}

\makeatletter
\let\c@equation\c@thm
\makeatother
\numberwithin{equation}{section}

\bibliographystyle{plain}

\voffset = -10pt
\headheight = 0pt
\topmargin = -20pt
\textheight = 690pt

%--------Meta Data: Fill in your info------
\title{18.781 \\
Problem Set 4.1}

\author{John Wang}

\begin{document}

\maketitle

\section{Problem 1}

\begin{prob}
Show that the only cube root of 1 modulo 1024 is 1. 
\end{prob}

\begin{sol}
We want to solve the equation $x^3 \equiv 1 \pmod{1024}$. We will use Hansel's lemma and note that $1024 = 2^{10}$. Thus, we can find solutions to $f(x) = x^3 - 1$ in the congruence $f(x) \equiv 0 \pmod{2}$ and lift solutions to $f(x) \equiv 0 \pmod{1024}$. We see that $a_1 = 1$ is the only solution to $f(x) \equiv 0 \pmod{2}$ by inspection. Thus, we find that $f'(a) = 3(1)^2 \equiv 1 \pmod{2}$. Thus, we see that $\overline{f'(a)} = 1 \pmod{2}$. Using Hansel's lemma, we get:

\begin{eqnarray}
a_2 &\equiv& a_1 - f(a_1) \overline{f'(a)} \pmod {2^2} \\
&\equiv& 1 - (0)(1) \pmod{2^2} \\
&\equiv& 1 \pmod {2^2} \\
a_3 &\equiv& 1 - (0)(1) \pmod{2^3} \\
&\equiv& 1 \pmod{2^3} \\
&\vdots&
\end{eqnarray}

It is clear that $f(a_i) \equiv 0 \pmod{2^i}$ for all $i$ by induction. Therefore, since $a_{i+1} \equiv a_i - f(a_i) \overline{f'(a)} \pmod{2^i} \equiv a_i \pmod{2^i}$ for all $i$, we know that $a_i \equiv 1 \pmod{2^i}$ for all $i$. By Hansel's lemma, there are no other solutions, so $x = 1 \pmod{1024}$ is the only solution to $f(x) \equiv 0 \pmod{1024}$. 
\end{sol}

\begin{prob}
Find all the cube roots of $-3$ modulo $1024$. 
\end{prob}

\begin{sol}
We want to solve $f(x) \equiv 0 \pmod{1024} \equiv 0 \pmod{2^{10}}$ where $f(x) \equiv x^3 - 3$. First, we note that there is no solution for $f(x) \equiv 0 \pmod{2}$ and small powers of $2$. A solution of $a_1 = 5$ occurs for $f(x) \equiv -3 \pmod{8}$. Using Hansel's lemma, we see that $a_2 \equiv 5 - (5^3 + 3) 3 \pmod{16} \equiv -379 \pmod{16} \equiv 5 \pmod{16}$. If we keep going, we get:
\begin{eqnarray}
a_3 &\equiv& 5 \pmod {32} \\
a_4 &\equiv& 5 - (5^3 + 3) 3 \pmod{64} \equiv 5 \pmod{64} \\
a_5 &\equiv& 5 - (5^3 + 3) 3 \pmod{128} \equiv 5 \pmod{128} \\
a_6 &\equiv& 5 - (5^3 + 3) 3 \pmod{256} \equiv 133 \pmod{256} \\
a_7 &\equiv& 133 - (133^3 + 3) 3 \pmod{512} \equiv 133 \pmod{512} \\
a_8 &\equiv& 133 - (133^3 + 3) 3 \pmod{1024} \equiv 645 \pmod{1024}
\end{eqnarray}

There are no other solutions to $f(x) \equiv 0 \pmod{1024}$, thus the only solution is $x = 645 \pmod{1024}$. 
\end{sol}

\begin{prob}
Solve $x^5 + x^4 + 1 \equiv 0 \pmod{3^4}$. 
\end{prob}

\begin{sol}
First, note that Hansel's lemma fails since $f(x) = x^5 + x^4 + 1$ and $f'(x) = 5 x^4 + 4x^3$. This means, since $a_1 = 1$ for $f(x) \pmod{3}$, we have $f'(a_1) = 5 + 4 = 9 \equiv 0 \pmod{3}$. However, we can use the construction of Hansel's lemma to attempt to proceed. We see that we must find $t$ such that $f(a + 3t) \equiv 0 \pmod{3^3}$. By the proof of Hansel's lemma, we know that a unique $t$ occurs when $f(a) + 3t f'(a) \equiv 0 \pmod{3^2}$. Since we have already seen that $f'(a_1) = 0$, we have $f(a) \equiv 0 \pmod{3^2}$. However, since $f(a) = (1)^5 + (1)^4 + 1 = 3 \not \equiv 0 \pmod{3^2}$, we cannot have any solutions to $f(x) \pmod{3^2}$. The same logic follows for $f(x) \pmod{3^3}$ and $f(x) \pmod{3^4}$. There there are no solutions to the congruence.
\end{sol}

\section{Problem 2}
\begin{prob}
Write a gp program to implement Pollard rho. Use it to find a prime factor of $2^{1231} - 1$.
\end{prob}

\begin{sol}
The Pollard rho algorithm implemented in gp uses a helper function $f(x,n)$ which just computes $f(x,n) = x^2 + 1 \pmod{n}$. The entire code is given below:
\begin{verbatim}    
    f(x,n)=(x^2 + 1) % n;

    pr(n)=
    {
        x=2;
        y=5;
        d=1;
        while(d==1,
            x=f(x,n);
            y=f(f(y,n),n);
            d=gcd(abs(y-x),n);
        );
        if (d==n, print("Failure"), print(d));
    }

\end{verbatim}

Using this algorithm on $2^{1231} - 1$, we obtain a prime factor of $p = 531,793$. 
\end{sol}

\section{Problem 3}

\begin{prob}
Suppose that $N = pq$ is the product of two primes. Suppose in addition to knowing $N$, we also know $M = \phi(N)$. Describe how you would obtain $p$ and $q$ from this information. 
\end{prob}

\begin{sol}
First, we know that $\phi(N) = \phi(p) \phi(q)$ because $p$ and $q$ are prime factors of $N$. Moreover, since $p,q$ are primes, we know that $\phi(p) = p - 1$ and $\phi(q) = q - 1$. Thus, we see that $M = \phi(N) = (p-1) (q-1) = pq - p - q + 1 = N - p - q + 1$. From this, we see that $p = N - M - q + 1$. Since $pq = N$, we see that the following is true:
\begin{eqnarray}
N &=& pq = (N - M - q + 1) q \\
0 &=& q^2 + q (M - N) + N - 1
\end{eqnarray}

This is a quadratic in $q$, which can be solved using the quadratic formula:
\begin{eqnarray}
q = \frac{-(M- N) \pm \sqrt{ (M-N)^2 - 4(N-1)}}{2}
\end{eqnarray}

Then, we can use $q$ to figure out $p$ by using $p = N/q$. However, notice that $p$ will  simply be the other solution of the quadratic since $p$ and $q$ are symmetric so their quadratic equations will be the same.  

\end{sol}

\begin{prob}
Use the above method to factor the number:
\begin{equation}
N = 27606985387162255149739023449107931668458716142620601169954803000803329
\end{equation}

which is a product of two primes given that:
\begin{equation}
\phi(N) = 27606985387162255149739023449107761527112996396559656119259509106409476
\end{equation}
\end{prob}

\begin{sol}
We will run the above algorithm. Solving the equation, we obtain:
\begin{eqnarray}
q &=& 162259276829213363391578010288127 \\
p &=& 170141183460469231731687303715884105727
\end{eqnarray}

A check check using gp shows that both $p$ and $q$ are prime numbers, and that $pq = N$. 
\end{sol}

\section{Problem 4}

\begin{prob}
Suppose that $f(x) \equiv 0 \pmod{p^j}$ and that $f'(a) \not \equiv 0 \pmod{p}$. Let $\overline{f'(a)}$ be an integer chosen so that $f'(a) \overline{f'(a)} \equiv 1 \pmod{ p^{2j}}$ and set $b= a - f(a) \overline{f'(a)}$. Show that $f(b) \equiv 0 \pmod{p^{2j}}$.
\end{prob}

\begin{sol}
First, we will use a small lemma, which is very close to the lemma proven in class:
\begin{lem}
If $j \geq 1$ then $f(a + t p^{j}) = f(a) + t p^{j} f'(a) \pmod{p^{2j}}$.
\end{lem}
\begin{proof}
We will use a Taylor expansion about $a$ to find that:
\begin{eqnarray}
f(a + t p^{j}) &=& f(a) + t p^{j} f'(a) + (t p^{j})^2 \frac{f''(a)}{2!} + \ldots \\
&=& f(a) + t p^{j} f'(a) \pmod{2^{2j}}
\end{eqnarray}

The second line follows because $f''(a)/2! = (k)(k-1)a^{k-2}/2 = {k \choose 2} a^{k - 2} \in \mathrm{Z}$. Moreover, each term $f^{(r)}(a) / r!$ is an integer because $f^{(r)}(a)/r! = {k \choose r} a^{k - r}$ using the logic from above. Therefore, we see that all these terms are integers, and that $p^{2j}, p^{3j}, p^{4j}, \ldots$ are all divisible by $p^{2j}$. Therefore, the terms:
\begin{eqnarray}
(tp^j)^{2} \frac{f''(a)}{2!} + (tp^{j})^3 \frac{ f'''(a)}{3!} + \ldots \equiv 0 \pmod{2^{j}}
\end{eqnarray}

This completes the proof of the lemma.
\end{proof}

Now that we have this result, we want to find a $t$ such that $0 \equiv f(a) + t p^{j} f'(a) \pmod{p^{2j}}$. Rearranging terms, we can solve for $t$ and we find that $t \equiv -\frac{f(a)}{p^{j}} \overline{f'(a)} \pmod{2j}$. We know that this is an integer because $f(a) \equiv 0 \pmod{p^{j}}$ so that $p^{j} | f(a)$ and so $\frac{f(a)}{p^j}$ is an integer. Moreover, we see the following with this result:
\begin{eqnarray}
a + tp^j = a + \frac{f(a)}{p^j} \overline{f'(a)} p^j = a + f(a) \overline{f'(a)} = b
\end{eqnarray}

Thus, we have shown that $b$ satisfies $f(b) \equiv 0 \pmod{p^{2j}}$. 
\end{sol}

\section{problem 5}

\begin{prob}
Let $p$ be a prime. Let $\sigma_1, \sigma_2, \ldots, \sigma_{p-1}$ be the elementary symmetric polynomials in $1,2, \ldots, p-1$ as in class (i.e. $\sigma_k$ is the sum of products of $k$ of these numbers). We showed that $(-1)^{p-1} \sigma_{p-1} = (p-1)! \equiv -1 \pmod{p}$. Show that $\sigma_{1}, \ldots, \sigma_{p-2}$ are all congruent to $0 \pmod{p}$.
\end{prob}

\begin{sol}
We showed in class that the following congruence must hold:
\begin{eqnarray}
(x-1)(x-2)\ldots (x - (p-1)) \equiv x^{p-1} - 1 \pmod{p}
\end{eqnarray}

However, we also showed in class that if we have a product $f(x) = (x - \alpha_1)\ldots (x - \alpha_p)$ where $\alpha_i \in \{1,2,\ldots, p-1\}$, then we have $f(x) = x^{p-1} - \sigma_1 x^{p-2} + \sigma_2 x^{p-3} + \ldots + (-1)^{p-1} \sigma_{p-1}$. Thus, the following congruence must hold:
\begin{eqnarray}
x^{p-1} - \sigma_1 x^{p-2} + \sigma_2 x^{p-3} + \ldots + (-1)^{p-1} \sigma_{p-1} \equiv x^{p-1} - 1 \pmod{p} 
\end{eqnarray}
Since we know that the constant term on the left hand side is $(-1)^{p-1} \sigma_{p-1} \equiv -1 \pmod{p}$, we can subtract $x^{p-1} - 1$ from both sides. Using this observation, we obtain:
\begin{eqnarray}
- \sigma_1 x^{p-2} + \sigma_2 x^{p-3} + \ldots + (-1)^{p-2} \sigma_{p-2} x  \equiv 0 \pmod{p}
\end{eqnarray}

However, each one of these terms is a different power of $x$. Since this congruence must hold for all $x$, we know that the coefficients on each of the $x^{i}$ terms must be equivalent to $0 \pmod{p}$. This shows that $\sigma_1, \ldots, \sigma_{p-2}$ are all congruent to $0 \pmod{p}$. 
\end{sol}

\begin{prob}
For $p \geq 5$, show that $\sigma_{p-2} \equiv 0 \pmod{p^2}$. 
\end{prob}

\begin{sol}
As we noted before, we know that the following equation holds from lecture:
\begin{eqnarray}
(x-1)(x-2)\ldots(x-(p-1)) = x^{p-1} - \sigma_1 x^{p-2} + \ldots + \sigma_{p-1}
\end{eqnarray}

Now, we let $x = p$, and we find the following:
\begin{eqnarray}
(p-1)(p-2)\ldots(p-(p-1)) &=& p^{p-1} - \sigma_1 p^{p-2} + \ldots + \sigma_{p-1} \\
(p-1)! &=&  p^{p-1} - \sigma_1 p^{p-2} + \ldots + \sigma_{p-1} 
\end{eqnarray}

Since we know that $(-1)^{p-1} \sigma_{p-1} = (p-1)!$ from lecture and from the statement of the problem, we see that we can rearrange the above expression and cancel out $\sigma_{p-1}$ (since $(-1)^{p-1} = 1$ as $p-1$ is even). This gives:
\begin{eqnarray}
0 = -\sigma_1 p^{p-2} + \sigma_2 p^{p-3} + \ldots  - \sigma_{p-3} p^2 + \sigma_{p-2} p \\
\end{eqnarray}

Since $\sigma_{i} \in \mathrm{Z}$, if we take everything modulo $p^2$, we notice that $p^2 | p^{i}$ for all $i \geq 2$. These two facts imply that $(-1)^{i} \sigma_{i+1} p^{i} \equiv 0 \pmod{p^2}$ for all $i > 2$. This implies that:
\begin{eqnarray}
\sigma_{p-2} \equiv 0 \pmod{p^2}
\end{eqnarray}

This completes the proof.
\end{sol}

\section{Problem 6}

\begin{prob}
Let $p$ be a prime, and $g$ a primitive root modulo $p$. Show that $1,g,g^2, \ldots, g^{p-2}$ are all the nonzero residue classes mod $p$.
\end{prob}

\begin{sol}
First, since $g$ is a primitive root modulo $p$, we know that $ord_p(g) = p-1$. Therefore, we see that $p-1$ is the lowest power $k > 0$ such that $g^k \equiv 1 \pmod{p}$. If we can show that $1,g,g^2, \ldots, g^{p-2}$ are all distinct modulo $p$, then we know that they constitute all the non-zero residue classes modulo $p$. Suppose not. Then there exist integers $i$ and $j$ such that $0 \leq i \neq j \leq p - 2$ such that $g^i \equiv g^j \pmod{p}$. Without loss of generality, assume that $i > j$. We see that this implies $g^{i - j} \equiv 1 \pmod{p}$ using division by the greatest common divisor. However, we know that $i - j < p - 1$. However, we know that $ord_p(g) = p - 1$, which is a contradiction because $i-j$ is a smaller power which satisfies $g^k \equiv 1 \pmod{p}$. 
\end{sol}

\begin{prob}
For a positive integer $k$, let $S_k = 1^k + 2^k + \ldots + (p-1)^k$. Compute the value of $S_k$ modulo $p$ in closed form as a function of $k$.
\end{prob}

\begin{sol}
First, we note that if $k \equiv p - 1 \pmod{p}$, then we must have $i^{k} \equiv i^{\phi(p)} \equiv 1 \pmod{p}$ for all $i \in \{1,2,\ldots, p-1\}$ using Euler's Theorem. Since there are $p-1$ of these terms in the sum, we find that $S_{p-1} \equiv p-1 \pmod{p}$. 

Now suppose that $k$ is not a multiple of $p-1$. Then let $g$ be a primitive root of modulo $p$, which we know exists because $p$ is prime. Using the result that we have proven above, we know that $1,g,g^2, \ldots, g^{p-2}$ are all the nonzero residue classes modulo $p$. Thus, we know that the above set is just a reordering of $1,2,3, \ldots, p-1$. This means we can write:
\begin{eqnarray}
\sum_{i=1}^{p-1} i^k &\equiv& \sum_{i=1}^{p-1} (g^i)^k \pmod{p} \\
&\equiv& \sum_{i=1}^{p-1} (g^k)^i \pmod{p} \\
&\equiv& \frac{g^k ((g^{k})^{p-1} - 1)}{g^k - 1} \pmod{p}
\end{eqnarray}

Where the last line comes from the expression for a finite geometric series. However, since we know that $(g^k)^{p-1} \equiv 1 \pmod{p}$ by Fermat's Little Theorem, we know that the entire sum comes out to zero. Moreover, we know that the denominator is not zero because $p - 1 \nmid k$ by assumption, so that $g^{k} \not \equiv 1 \pmod{p}$. This shows that $S_k = 0$ for all $k$ such that $p-1 \nmid k$. We then have the following formula:
\begin{equation}
S_k \equiv \left\{ \begin{array} {l l}
p-1 \pmod{p} & \text{if } k \equiv 0 \pmod{p} \\
0 \pmod{p} & \text{otherwise}
\end{array} \right. 
\end{equation}
\end{sol}
\end{document}