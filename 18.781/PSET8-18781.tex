\documentclass[psamsfonts]{amsart}

%-------Packages---------
\usepackage{amssymb,amsfonts}
\usepackage[all,arc]{xy}
\usepackage{enumerate}
\usepackage{mathrsfs}
\usepackage[margin=1in]{geometry}
\usepackage{thmtools}
\usepackage{verbatim}
\usepackage{multirow}


%--------Theorem Environments--------
%theoremstyle{plain} --- default
\newtheorem{prob}{Problem}[section]
\newtheorem{thm}{Theorem}[section]
\newtheorem{cor}[thm]{Corollary}
\newtheorem{prop}[thm]{Proposition}
\newtheorem{lem}[thm]{Lemma}
\newtheorem{conj}[thm]{Conjecture}
\newtheorem{quest}[thm]{Question}

\newenvironment{sol}{{\bfseries Solution}}{\qedsymbol}

\def\legendre(#1,#2){%
{#1 \overwithdelims () #2} }

\theoremstyle{definition}
\newtheorem{defn}[thm]{Definition}
\newtheorem{defns}[thm]{Definitions}
\newtheorem{con}[thm]{Construction}
\newtheorem{exmp}[thm]{Example}
\newtheorem{exmps}[thm]{Examples}
\newtheorem{notn}[thm]{Notation}
\newtheorem{notns}[thm]{Notations}
\newtheorem{addm}[thm]{Addendum}
\newtheorem{exer}[thm]{Exercise}

\theoremstyle{remark}
\newtheorem{rem}[thm]{Remark}
\newtheorem{rems}[thm]{Remarks}
\newtheorem{warn}[thm]{Warning}
\newtheorem{sch}[thm]{Scholium}


\makeatletter
\let\c@equation\c@thm
\makeatother
\numberwithin{equation}{section}

\bibliographystyle{plain}

\voffset = -10pt
\headheight = 0pt
\topmargin = -20pt
\textheight = 690pt

%--------Meta Data: Fill in your info------
\title{18.781 \\
Problem Set 7}

\author{John Wang}

\begin{document}

\maketitle

\section{Problem 1}

\begin{prob}
Let $n$ be a positive integer. Evaluate $\sum_{k=0}^n {n \choose 3k}$. 
\end{prob}

\begin{sol}
First, we know that $(1 + x)^n = \sum_{k=0}^{\infty} {n \choose k} x^k$ is the generating function for the expression. Let us now use the third roots of unity $\omega$ and substitute them into the expression for the generating function. We know that the third roots of unity are $1$ and $-\frac{1}{2} \pm\frac{ i\sqrt{3}}{2}$. Moreover, we know that the following holds, using the fact that $\omega^2 = -\omega - 1$:
\begin{eqnarray}
\sum_{k=0}^{n} \omega^k {n \choose k} &=& (1 + \omega)^n = (- \omega^2)^n \\
\sum_{k=0}^{n} \omega^{2k} {n \choose k} &=& (1 + \omega^2)^n = (- \omega)^n \\
\sum_{k=0}^{n} 1^k {n \choose k} &=& (1 + 1)^n = 2^n
\end{eqnarray}

Therefore, since we know that summing up each of these series, we will obtain the sum $3 \sum_{k=0}^n {n \choose 3k}$, we find that:
\begin{eqnarray}
3 \sum_{k=0}^n {n \choose 3k} = 2^n + (-\omega)^n + (- \omega)^{2n} \\
\sum_{k=0}^n {n \choose 3k} =  \frac{1}{3} \left( 2^n + (-\omega)^n + (- \omega)^{2n} \right)
\end{eqnarray}
The completes the calculation. 
\end{sol}

\section{Problem 2}

For a sequence $\{ a_n \}$ we can define another kind of generating function called an exponential generating function as follows: 
\begin{equation}
\bar{A}(x) = \sum_{n \geq 0} a_n \frac{x^n}{n!}.
\end{equation}

It satisfies many of the nice properties we have seen for generating functions (for example, linearity with respect to the sequence). But some properties are slightly modified. 

\begin{prob}
Show that the generating function for the left-shifted sequence $\{a_1, a_2, \ldots\}$ is $\frac{d}{dx} \bar{A}(x)$. 
\end{prob}

\begin{sol}
We see that if we differentiate $\bar{A}(x)$, we obtain the following:
\begin{eqnarray}
\frac{d}{dx} \bar{A}(x) &=& \sum_{n \geq 0} a_n \frac{d}{dx} \frac{x^n}{n!} \\
&=& \sum_{n \geq 0} a_n \frac{n x^{n-1}}{n!} \\
&=& \sum_{n \geq 0} a_n \frac{x^{n-1}}{(n-1)!} \\
&=& \sum_{n \geq 1} a_n \frac{x^{n}}{n!} 
\end{eqnarray}

This is exactly what we wanted to show because the bottom expression is the sequence $\{a_n \}$ left shifted by one unit.
\end{sol}

\begin{prob}
If $\bar{A}(x)$ is the exponential generating function for $\{ a_n \}$ and $\bar{B}(x)$ is the exponential generating function for $\{b_n\}$, show that $\bar{A}(x) \bar{B}(x)$ is the exponential generating function for the sequence $\{c_n \}$ given by $c_n = \sum_{k=0}^n {n \choose k} a_k b_{n-k}$. 
\end{prob}

\begin{sol}
We shall compute $\bar{A}(x) \bar{B}(x)$ by examining the coefficients of $\frac{x^n}{n!}$ for each $n \geq 0$. We see that if we multiply the terms on $a_k$ and $b_{n-k}$ we obtain:
\begin{eqnarray}
a_k \frac{x^k}{k!} b^{n-k} \frac{x^{n-k}}{(n-k)!} &=& a_k b_{n-k} \frac{x^n}{k! (n-k!)}
\end{eqnarray}

Moreover, we know that the only terms that have $x^n$ are those where $k + (n-k) = n$. This includes all $k$ ranging from $0$ to $n$. However, we know that in the exponential generating function, we must have $\frac{x^n}{n!}$ as the term without the coefficient $c_n$. However, we know that $\frac{x^n}{n!} {n \choose k} = \frac{x^n}{k! (n - k)!}$. Therefore, we know 
\begin{eqnarray}
a_k b_{n-k} \frac{x^n}{k!(n-k)!} &=& {n \choose k} a_k b_{n-k}  \frac{x^n}{n!}
\end{eqnarray} 

Since we range over all $k \in \{0, 1, \ldots, n\}$, we see that $c_n = \sum_{k=0}^n {n \choose k} a_k b_{n-k}$ as desired.
\end{sol}

\begin{prob}
Show that the generating function $E(x)$ for the sequence $a_n = r^n$ (where $r$ is some fixed complex number) satisfies $E'(x) = r E(x)$. Solve this differential equation to deduce that $e^{rx} = \sum_{n \geq 0} \frac{r^n x^n}{n!}$. 
\end{prob}

\begin{sol}
We know that $E(x) = 1 + rx + r^2 + \frac{x^2}{2!} + \ldots  = \sum_{n \geq 0} r^n \frac{x^n}{n!}$. Taking the derivative we find:
\begin{eqnarray}
E'(x) &=& \sum_{n \geq 0} r^n \frac{d}{dx} \frac{x^n}{n!} \\
&=& \sum_{n \geq 1} r^n \frac{x^{n-1}}{(n-1)!} \\
&=& r \sum_{n \geq 0} r^{n} \frac{x^{n}}{n!} \\
&=& r E(x)
\end{eqnarray}

This is an easily solvable differential equation. We know that $E(x) = e^{rx}$ by elementary differential equations class. This means that $e^{rx} = \sum_{n \geq 0} \frac{r^n x^n}{n!}$. 
\end{sol}

\section{Problem 3}

Define a sequence $B_n$ by the identity $f(x) := \frac{x}{e^x - 1} = \sum_{n=0}^\infty \frac{B_n}{n!} x^n$, i.e. $B_n$ is $n!$ times the coefficient of $x^n$ in the expansion of the left hand side, where one uses $e^{x} - 1 = \sum_{n=0}^\infty x^n/ n!$. 

\begin{prob}
Calculate $B_0$ through $B_{10}$. 
\end{prob}

\begin{sol}
First, we can clearly see from the series expansion of the left hand side that $B_0 = 1$. Next, we see that if we multiply both sides of the definition of $f(x)$ by $(e^x - 1)$ and expand out the series, we obtain $x = (e^x - 1) \sum_{n=0}^\infty \frac{B_n}{n!} x^n$. Using the fact that $\bar{A}(x) \bar{B}(x)$ generates the function $\{c_n \}$ where $c_n = \sum_{k=0}^n {n \choose k} a_k b_{n-k}$, we find:
\begin{eqnarray}
x &=& \left( \sum_{n=0}^\infty (1) \frac{x^n}{n!} \right) \left( \sum_{n=0}^\infty B_n \frac{x^n}{n!} \right) \\
x &=& \sum_{n=0}^\infty \sum_{k=0}^n {n \choose k } B_k 
\end{eqnarray}

Equating the coefficients of $x^{n+1}$ on both sides of the expression, we find that:
\begin{eqnarray}
0 &=& \sum_{k=0}^{n+1} {n + 1 \choose k } B_k \\
(n+1) B_n &=& - \sum_{k=0}^{n-1} {n + 1 \choose k} B_k
\end{eqnarray}

This means that we can use this recurrence to calculate $B_0, \ldots, B_{10}$. We obtain:

\begin{eqnarray}
B_0 &=& 1 \\
B_1 &=& - \frac{1}{2} \\
 B_2 &=& \frac{1}{6} \\
 B_3 &=& 0 \\
 B_4 &=& - \frac{1}{30} \\
B_5 &=& 0 \\
B_6 &=& \frac{1}{42} \\
B_7 &=& 0 \\
B_8 &=& - \frac{1}{30} \\
B_9 &=& 0 \\
B_{10} &=& \frac{55}{66} 
\end{eqnarray}
\end{sol}

\begin{prob}
Show that for $n > 1$ odd, $B_n = 0$. 
\end{prob}

\begin{sol}
We see that $f(x) - f(-x)$ can be expanded into the following, using the series representations:
\begin{eqnarray}
f(x) - f(-x) &=& \sum_{n=0}^\infty B_n \frac{x^n}{n!} - \sum_{n=0}^{\infty} B_n \frac{(-x)^n}{n!} \\
&=& \sum_{n \textrm{ odd}} B_n \frac{x^n}{n!} 
\end{eqnarray}

Moreover, we know that $f(x) - f(-x)$ can also be represented as:
\begin{eqnarray}
f(x) - f(-x) &=& \frac{x}{e^x - 1} - \frac{-x}{e^{-x} - 1} \\
&=& \frac{x(e^{-x} - 1) + x(e^{x} - 1)}{(e^{x} - 1)(e^{-x} - 1)} \\
&=& \frac{x e^{-x} + x e^{x} - 2x}{2 - e^x - e^{-x}} \\
&=& -x 
\end{eqnarray}

Thus, equating these two expressions, we see that $B_1 = -1$, and that $B_n = 0$ for all odd $n$. 
\end{sol}

\begin{prob}
Establish the recurrence for $n \geq 2$ that $\sum_{k=0}^{n-1} {n \choose k} B_k = 0$. 
\end{prob}

\begin{sol}
We use the same method we used in part 1 and multiply both sides by $e^x - 1$. Thus, we find:
\begin{eqnarray}
x &=& \left( \sum_{n=0}^\infty (1) \frac{x^n}{n!} \right) \left( \sum_{n=0}^\infty B_n \frac{x^n}{n!} \right) \\
x &=& \sum_{n=0}^\infty \sum_{k=0}^n {n \choose k } B_k 
\end{eqnarray}

Finding the coefficient of $x^n$ on both sides, we see that $B_n = \sum_{k=0}^n {n \choose k} B_k$, which shows, since ${n \choose n} B_n = B_n$, that we have:
\begin{eqnarray}
0 = \sum_{k=0}^{n-1} {n \choose k} B_k 
\end{eqnarray}

This is what we wanted to show.
\end{sol}

\begin{prob}
Let $S_k(n) = 1^k + 2^k + \ldots + n^k$ be the sum of the $k$th powers of the first $n$ natural numbers. Show that $S_k$ is given by the following polynomial of degree $k+1$ in $n$: $S_k(n) = \frac{1}{k+1} \sum_{i=0}^k (-1)^i {k + 1 \choose i} B_i n^{k+1 - i}$. 
\end{prob}

\begin{sol}
First, we know that $\sum_{k \geq 0} S_k(n) \frac{x^k}{k!}$ can be computed as follows:
\begin{eqnarray}
\sum_{k \geq 0} S_k(n) \frac{x^k }{k!} &=& \sum_{k=0}^\infty \left( \sum_{m=0}^n k^m \right) \frac{x^k}{k!} \\
&=& \sum_{m=0}^n \sum_{k=0}^\infty \frac{k^m x^k }{k!} \\
&=& \sum_{m=0}^n e^{mx}
\end{eqnarray}

Where the last line comes from the result we showed in problem 2. However, we know that $\sum_{m=0}^\infty e^{mx}$ is a geometric series which can be evaluted. We find:
\begin{eqnarray}
\sum_{m=0}^n e^{mx} &=& \frac{e^{nx} - 1}{e^x - 1} \\
&=& \frac{e^{nx} - 1}{x} \frac{x}{e^{x} - 1} \\
&=& \left( \sum_{k=0}^\infty \frac{n^{k+1}}{k+1} \frac{x^k}{k!} \right) \left( \sum_{k=0}^\infty B_k \frac{x^k}{k!} \right)
\end{eqnarray}

And now we use result from problem 2 on the product of two exponential generating functions to continue this:
\begin{eqnarray}
\sum_{k \geq 0} S_k(n) \frac{x^k}{k!} = \sum_{m=0}^n e^{mx} = \sum_{k=0}^\infty \left( \sum_{i=0}^k {k \choose i} \frac{1}{k-i+1} B_i n^{k+1-i} \right) \frac{x^k}{k!}
\end{eqnarray}

Now, we can equate the coefficients of $\frac{x^k}{k!}$ on the left and right hand sides. We find:
\begin{eqnarray}
S_k(n) &=& \sum_{i=0}^k {k \choose i} \frac{1}{k-i+1} B_i n^{k+1-i} \\
&=& \sum_{i=0}^k \frac{k!}{i!(k-i)!(k-i+1)} B_i n^{k+1-i} \\
&=& \sum_{i=0}^k \frac{1}{k+1} {k+1 \choose i} B_i n^{k+1-i} \\
&=& \frac{1}{k+1} \sum_{i=0}^k (-1)^i { k + 1 \choose i} B_i n^{k+1 - i} 
\end{eqnarray}

Which is what we wanted to show.
\end{sol}

\section{Problem 4}

\begin{prob}
Let $m$ and $n$ be two integers which are sums of two squares. Show that $mn$ is a sum of two squares. Use this to show that any positive integer of the form $\prod p_i^{e_i} \prod q_j^{f_j}$, where $p_i$ are primes which are 2 or 1 mod 4, and $q_j$ are primes which are 3 mod 4, such that $f_j$ are all even, is a sum of two integer squares. 
\end{prob}

\begin{sol}
First, if $m = a^2 + b^2$ and $n = c^2 + d^2$, then it is clear that $mn$ is also a sum of two squares because we can write $mn = (a^2 + b^2)(c^2 + d^2) = (ac)^2 + (ad)^2 + (bc)^2 + (bd)^2 = (ac + bd)^2 + (ad - bc)^2$. This shows that $mn$ is a sum of two squares. 

Now, note that if $p$ is a prime such that $p =2$ or $p \equiv 1 \pmod{4}$, then $p$ is a sum of two squares by a theorem proven in class. This means that $\prod p_i^{e_i}$ is a sum of two squares, since $p_i^{e_i}$ are all sum of two squares and we can therefore apply our previous result. 

Moreover, since $q_j$ are all primes congruent to $3 \pmod{4}$ where $f_j$ are all even, we know that $q_j^{f_j} = (q_j^{f_j/2})^2 + 0^2$ is a sum of two squares. This, all of the $q_j$ are sums of two squares, which means that $\prod q_j^{f_j}$ is a sum of two squares. Therefore, we see that $\prod p_i^{e_i} \prod q_j^{f_j}$ is a product of two squares given the assumptions of the problem. 
\end{sol}

\begin{prob}
Now suppose $n$ is a sum of two integer squares. Show that it must have the form above, i.e. if a prime $q$ which is 3 mod 4 divides n, then it must divide it to an even power. 
\end{prob}

\begin{sol}
We note that this is equivalent to proving that if $q$ is a prime of the form $3 \pmod{4}$, then it does not divide a sum of two coprime squares. Assume by contradiction that $q | x^2 + y^2$ where $gcd(x,y) = 1$. Thus, we see that $x^2 + y^2 \equiv 0 \pmod{q}$. Since we can choose $x$ and $y$ mod $q$ such that $-q/2 < x,y < q/2$, we can set $x^2 + y^2 < (1/2) q^2$. Since we can write
\begin{eqnarray}
\left( \frac{x+y}{2} \right)^2 + \left( \frac{x - y}{2} \right)^2 = \frac{x^2 + y^2}{2}
\end{eqnarray}

We can substitute $a = (x+y)/2$ and $b = (x-y)/2$ and find that $qs = a^2 + b^2$ where $s$ is an odd number (since there are no factors of 2 left in $a$ or $b$). Moreover, we know that $s < q$ since $x^2 + y^2 < (1/2) q^2$. Since $a$ and $b$ are coprime, $a^2 + b^2 \equiv 1 \pmod{4}$. This means that $s \equiv 3 \pmod {4}$ so that $s$ must have at least one prime factor $p \equiv 3 \pmod{4}$. Therefore, if $q \equiv 3 \pmod{4}$ divides a sum of two coprime squares, then $p \equiv 3 \pmod{4}$ less than $q$ must also divide this sum of two coprime squares. However, we assumed that these squares were coprime, which is a contradiction. Therefore, $q \equiv 3 \pmod{4}$ must divide $n$ to an even power if $n$ is a sum of two integer squares.
\end{sol}

\begin{prob}
Show that $n$ is a sum of squares of two rational numbers if and only if it's a sum of squares of two integers.
\end{prob}

\begin{sol}
It is clear that if $n$ is a sum of squares of two integers, then it is also a sum of squares of two rational numbers. Thus, we only need to show that if $n$ is a sum of square of two rational numbers, then it is a sum of squares of two integers. Now suppose $n = (a/b)^2 + (c/d)^2$. We can rewrite this expression as $n(bd)^2 = (ad)^2 + (cb)^2$. Thus, we want to show that if there exist positive integers satisfying $m^2 n = x^2 + y^2$, then $n$ can be written as a sum of squares of two integers. 

Let $t$ be choosen so that $t^2 < n < (t+1)^2$. Since there are $(t+1)^2$ integers of the form $xu + yv$ where $0 \leq u$ and $v \leq t$, which is greater than $n$, we see that these numbers fill up a quadratic residue class modulo $n$. Thus, there are two numbers of the form $x (u - u') + y(v-v')$ which are divisible by $n$. We can set $w = u - u'$ and $z = v - v'$, knowing that $|w|, |z| < t$. This means that $n | x^2 w^2 - y^2 z^2$. Moreover, we know that $n | x^2 z^2 + y^2 z^2$. Therefore, $n$ divides the sum of these two, so that $n | x^2 ( w^2 + z^2)$. Since it is sufficient to show this when $gcd(x,n) =1$, we know that $n |  w^2 + z^2$. Since $x^2 + y^2 < 2n$, we see that $n = w^2 + y^2$. This completes the proof.
\end{sol}

\section{Problem 5}

\begin{prob}
Let $\omega = e^{2 \pi i/3}$ be a primitive cube root of unity. Write down the cyclotomic polynomial $\Phi_3 (x)$ and thereby compute $\omega^2$ in terms of $\omega$. Now calculate the norm of the complex number $a + b \omega$. Use this to show that if $m,n$ are two integers which can be written in the form $a^2 - ab + b^2$, then their product can also be written likewise. 
\end{prob}

\begin{sol}
Using the factorization of $x^3 - 1$ for all factors $x^k - 1$ where $k | 3$, we find that $\Phi_3(x) = x^2 + x + 1$. Substituting $x = \omega$ into the polynomial, we find that $0 = \omega^2 + \omega + 1$ so that $\omega^2 = - \omega - 1$. To find the norm of $a + b \omega$, we see that $(a + b \omega) (a + b \omega^2) =  a^2 + ab \omega^2 + ab \omega + b^2 \omega^3 =  a^2 + ab ( - 1 - \omega) + b^2  = a^2 - ab + b^2$. Thus, we see that $|a + b \omega| = a^2 - ab + b^2$. 

Now, we see that if an integer can be written in the form $a^2 - ab + b^2$, then it is the norm of a complex number $a + b \omega$. Therefore, if two integers $m,n$ can be written in the form $a^2 - ab + b^2$, then they are the norms of two numbers $a + b \omega$. Now, we know that the product $mn$ is the product of two norms. Since a product of two norms is also a norm, we know that $mn$ can be written in the form $a^2 - ab + b^2$. 
\end{sol}

\begin{prob}
Show that if $p$ is a prime which can be written as $a^2 - ab + b^2$, then $p$ cannot be $2 \pmod{3}$, so that $p= 3$ or $p \equiv 1 \pmod{3}$. 
\end{prob}

\begin{sol}
If $p = a^2 - ab + b^2$, then we can evaluate $a^2 - ab + b^2 \equiv p \pmod{3}$. We shall see what $p$ can be modulo 3. So we shall enumerate all the possibilities of $a,b \pmod{3}$ (i.e $a,b = \{0,1,2\}$) and observe what happens:
\begin{eqnarray}
0^2 - 0*0 + 0^2 &\equiv& 0 \pmod{3} \hspace{1cm} 0^2 - 0*1 + 1^2 \equiv 1 \pmod{3} \\
0^2 - 0*2 + 2^2 &\equiv& 1 \pmod{3} \hspace{1cm} 1^2 - 0*0 + 0^2 \equiv 1 \pmod{3} \\
1^2 - 1*1 + 1^2 &\equiv& 1 \pmod{3} \hspace{1cm} 1^2 - 1*2 + 2^2 \equiv 0 \pmod{3} \\
2^2 - 2*0 + 0^2 &\equiv& 1 \pmod{3} \hspace{1cm} 2^2 - 2*1 + 1^2 \equiv 0 \pmod{3} \\
2^2 - 2*2 + 2^2 &\equiv& 1 \pmod{3}
\end{eqnarray}

Thus, for all combinations of $a,b \pmod{3}$, it is impossible for $a^2 - ab + b^2$ to be equivalent to $2 \pmod{3}$. This completes the theorem.
\end{sol}

\section{Problem 6}

\begin{prob}
Show 3 and any prime $p$ which is $1 \pmod{3}$ can be written as $a^2 - ab + b^2$, for some integers $a$ and $b$. 
\end{prob}

\begin{sol}
If $p =3 $ then $a = 2$ and $b=1$ satisfies $p = 3 = a^2 - ab + b^2$. This means we can concern ourselves with all odd primes $p$ such that $p \equiv 1 \pmod{3}$. We shall use strong induction on the primes, so assume that every prime $q < p$ which is $1 \pmod{3}$ can be written in the form $a^2 -ab + b^2$. Next, we will invoke the following lemma:

\begin{lem}
There exists a positive integer $m < p$ such that $a^2 -ab +  b^2 = mp$. 
\end{lem}

\begin{proof}
We know from a theorem proven earlier in class that if $f(x) = ax^2 + bx + c$, then the congruence $f(x) \equiv 0 \pmod{p}$ has one solution if $p \nmid 1$ and $p | b^2 - 4ac$. So let $f(x) = x^2 - bx + b^2$. We see that the congruence has a solution when $p | b^2 - 4b^2 = - 3b^2$. Thus, the congruence has a solution when $p | 3b^2$. We can set $b = p$ so that this always occurs. Since this is the case, we see that $x=a$ and $b$ can always be found such that $p | a^2 - ab + b^2$ so that $mp = a^2 - ab + b^2$.  
\end{proof}

Now, if $m = 1$, then we are finished with the proof. If $m > 1$, we shall show that this results in a contradiction. So assume by contradiction that $m > 1$. We know that $a^2 - ab + b^2 < a^2 + b^2 < p^2$ because $m < p$. This implies that $|a| <p$ and $|b| < p$. We also know that $g = gcd(a,b) = 1$ because otherwise we could take $(a/g)^2 - (ab/g^2) + (b/g)^2 = (m/g^2) p$ and obtain a smaller $m$. 

Next, we know that $m$ must be odd. If not, then $a^2, ab, b^2$ must all be even (no other combination allows $mp$ to be even. This implies that $a$ and $b$ must have the same parity. Therefore, we can set $A = (2a + b)/2$ and $B= (a + b)/2$ and obtain:
\begin{eqnarray}
A^2 - AB + B^2 &=& \left( \frac{2a + b}{2} \right)^2 - \left( \frac{2a + b}{2} \frac{a + b}{2} \right) + \left( \frac{a+b}{2} \right)^2 \\
&=& \frac{3}{4} \left( a^2 - ab + b^2 \right)
\end{eqnarray}

This contradicticts the minimality of $m$, so we see that $m$ must be odd. Now, if we let $q$ be an odd prime dividing $m$ so that $m = q n$, we see that $a^2 - ab + b^2 = q n p$. We see that $q \nmid a$ and $q \nmid b$ or else $q$ would divide both $a$ and $b$ which would contradict the fact that $gcd(a,b) = 1$. Moreover, we know that $a^2 - ab + b^2 \equiv 1 \pmod{3}$ by the analysis given the previous problem. Thus, we see that $q p n \equiv 1 \pmod{3}$. Since $p \equiv 1 \pmod{3}$, we know that $qn \equiv 1 \pmod{3}$. Since we have shown that $n \equiv 1 \pmod{3}$ by the fact that $n$ is odd, we must have $q \equiv 1 \pmod{3}$. 

Because of this, we can use the inductive hypothesis, so we know that $q = c^2 - cd + d^2$. Thus, if we multiply both sides of $a^2 - ab + b^2 = q p n$ by $q$, we obtain:
\begin{eqnarray}
(a^2 - ab + b^2)(c^2 - cd + d^2) = p q^2 n
\end{eqnarray}

Simplifying this expression, we find that $(ac - bd)^2 + (ad + bc)^2 - (ac-bd)(ad+bc) = pq^2 n$. We see that $q | ac - bd$ and $q | ad + bc$ which implies:
\begin{eqnarray}
\left( \frac{ac - bd}{q} \right)^2 + \left( \frac{ad + bc}{q} \right)^2 + \left( \frac{ac - bd}{q} \frac{ad + bc}{q} \right) = pn
\end{eqnarray}

However, since we know that $n < m$ since $qn = m$, we see that we have contradicted the minimality of $m$. This completes the proof.
\end{sol}

\begin{prob}
An integer $n$ can be written as $a^2 - ab + b^2$ if and only if $n > 0$ and every prime $p \equiv 2 \pmod{3}$ which divides $n$, divides $n$ to an even power.
\end{prob}

\begin{sol}
In order to prove this theorem, we will invoke a lemma:
\begin{lem}
Let $p$ be a prime equivalent to $2 \pmod{3}$. Then no integer $n$ divisible precisely by an odd power of $p$ can be written in the form $a^2 - ab + b^2$.
\end{lem}

\begin{proof}
Let $p \equiv 2 \pmod{3}$ and suppose by contradiction that $n = a^2 - ab + b^2$ where $p^{2s + 1} || n$ for some integer $s$. Let $g = gcd(a,b)$ so that we can simplify the expression to the following: $n/g^2 = (a/g)^2 - (ab/g^2) + (b/g)^2$. Now, substitute $m = n/g^2$, $x = a/g$, and $y = b/g$ to obtain the expression
\begin{eqnarray}
m = x^2 - xy + y^2, \hspace{1cm} gcd(x,y) = 1
\end{eqnarray}

We see that $p^{2s+1} || m$, which means that $x^2 - xy + y^2 \equiv 0 \pmod{p}$. Now we can choose $A$ as an integer such that $2Ax = y \pmod{p}$. Substituting this expression for $y$, we find that $x^2 - x(2Ax) + (2Ax)^2 = x^2 - 2Ax^2 + 4A^2 x^2 = x^2 (4A^2 - 2A + 1) \equiv 0 \pmod{p}$. Completing the square and factorizing, this expression can be rewritten as $x^2 4^{-1} ((4A-1)^2 + 3) \equiv 0 \pmod{p}$. However, we know that $p \nmid x$ because if it did, then $p|y^2$ because $m = x^2 + y^2 - xy$. This would imply that $p |y$, and clearly $p$ does not divide both $x$ and $y$. Therefore, we see that $(4A - 1)^2 +3 \equiv 0 \pmod{p}$. 

Substituting $B = 4A - 1$, we find that $B^2 \equiv - 3 \pmod{p}$. However, we know that $\legendre(-3, p) = -1$ because $p \equiv 2 \pmod{3}$. This is a contradiction because we have shown that $B^2 \equiv - 3 \pmod{p}$. Therefore, we no integer $n$ divisible by an odd power of $p$ can be written in the form $a^2 - ab + b^2$ and we have completed the proof of the lemma.
\end{proof}

Now, let us go back to proving the original theorem and assume that $n$ can be written as $a^2 - ab + b^2$. By the lemma, we see that all primes $p \equiv 2 \pmod{3}$ which divide $n$ must have even powers when dividing $n$. This completes the first part of the proof.

For the converse, we let $n = p_1 p_2 \ldots p_m q_1^{2 e_1} \ldots q_l^{2 e_l}$ where $p_i \equiv 1 \pmod{3}$ and $q_i \equiv 2 \pmod{3}$. By part a of the problem, we see that $p_i$ can be written as $a^2 - ab + b^2$ for some integers $a$ and $b$. By a previous problem, we know that if $n$ and $m$ are integers that can be written in the form $a^2 - ab + b^2$, then their product can also be written in this form. This shows that the product $p_1 p_2 \ldots p_m$ can be written in the form $a^2 - ab + b^2$. We also know that $q_i^{2 e_i} = (q_i^{e_i})^{2} - (0) q_i^{e_i} + 0^2$ can be written in the form $a^2 - ab + b^2$. Thus, we see that $n$, which is the product of a elements which can be written in the form $a^2 - ab + b^2$, can also be written in the form $a^2 - ab + b^2$. This completes the proof of the converse.
\end{sol}

\section{Problem 7}

\begin{prob}
Calculate the continued fraction of $6157/783$. 
\end{prob}

\begin{sol}
We first find that $6157/783 = 7 + 676/783$. Thus, the first digit of the continued fraction is 7. Next, we want to find the continued fraction of $\frac{783}{676}$. This comes out to $1 + \frac{107}{676}$. Thus, the second digit of the original continued fraction is 1. Performing the same operations, we obtain the following series of operations:
\begin{eqnarray}
\frac{676}{107} &=& 6 + \frac{34}{107} \\
\frac{107}{34} &=& 3 + \frac{5}{34} \\
\frac{34}{5} &=& 6 + \frac{4}{5} \\
\frac{5}{4} &=& 1 + \frac{1}{4} 
\end{eqnarray}

Thus, the continued fraction of $6157/783$ comes out to $[7,1,6,3,6,1,4]$. 
\end{sol}

\begin{prob}
Calculate the continued fraction of $\sqrt{15}$. 
\end{prob}

\begin{sol}
The continued fraction of $\sqrt{15}$ can be calculated in the following manner, noting that $3 < \sqrt{15} < 4$:
\begin{eqnarray}
\sqrt{15} &=& 3 + \frac{\sqrt{15} -3}{1} = 3 + \frac{(\sqrt{15} - 3)(\sqrt{15} + 3)}{\sqrt{15} + 3} = 3 + \frac{6}{\sqrt{15} +3} \\
&=& 3 + \frac{1}{\frac{\sqrt{15}+3}{6}}
\end{eqnarray}

Now, we shall compute the continued fraction of $\frac{\sqrt{15} + 3}{6}$ to continue to find the continued fraction of $\sqrt{15}$:
\begin{eqnarray}
\frac{\sqrt{15} + 3}{6} &=& 1 + \frac{\sqrt{15} - 3}{6} = 1 + \frac{(\sqrt{15} - 3)(\sqrt{15}+3)}{6 (\sqrt{15} + 3)} \\
&=& 3 + \frac{1}{\sqrt{15} + 3}
\end{eqnarray}

Proceeding onwards, we will find the continued fraction of $\sqrt{15}+3$:
\begin{eqnarray}
\sqrt{15} + 3 &=& 6 + \frac{\sqrt{15} - 3}{1} = 6 + \frac{(\sqrt{15} -3)(\sqrt{15} + 3)}{\sqrt{15} +3} = 6 + \frac{6}{\sqrt{15}+3} \\
&=& 6 + \frac{1}{\frac{\sqrt{15}+3}{6}} 
\end{eqnarray}

We see that this has looped back to the continued fraction of $\frac{\sqrt{15}+3}{6}$. Therefore, we have found a recursive relationship, so that $\sqrt{15} = [3,1,6,1,6,\ldots]$. 
\end{sol}
\end{document}