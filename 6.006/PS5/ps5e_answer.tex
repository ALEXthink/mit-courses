%
% 6.006 problem set 5E
%
\documentclass[12pt,twoside]{article}

\usepackage{amsmath}
\usepackage{url}
\usepackage{mdwlist}
\usepackage{graphicx}
\usepackage{clrscode3e}
\newcommand{\isnotequal}{\mathrel{\scalebox{0.8}[1]{!}\hspace*{1pt}\scalebox{0.8}[1]{=}}}
\usepackage{listings}
\usepackage{tikz}
\usetikzlibrary{arrows}
\usetikzlibrary{matrix}
\usetikzlibrary{positioning}
\usetikzlibrary{shapes.geometric}
\usetikzlibrary{shapes.misc}
\usetikzlibrary{trees}

\setlength{\oddsidemargin}{0pt}
\setlength{\evensidemargin}{0pt}
\setlength{\textwidth}{6.5in}
\setlength{\topmargin}{0in}
\setlength{\textheight}{8.5in}

% Fill these in!
\newcommand{\theproblemsetnum}{5E}
\newcommand{\releasedate}{November 1, 2011}
\newcommand{\partaduedate}{Thursday, November 3}
\newcommand{\tabUnit}{3ex}
\newcommand{\tabT}{\hspace*{\tabUnit}}

\begin{document}

\handout{Problem Set \theproblemsetnum}{\releasedate}
Please download the .zip archive for this problem set, and refer to the
\texttt{README.txt} file for instructions on preparing your solutions.

You will need to submit the grading explanation by \textbf{Thursday, November
3rd, 11:59PM}. Your grade for Problem 1(p) in Problem Set 5 will be based on
both your solutions and the grading explanation.
\medskip

\hrulefill

\begin{problems}

\problem Please write a short explanation of your pseudo-code for Problem 1(p)
in Problem Set 5. Assume the grader is familiar with the problem and with the
staff solution. If your solution is very similar to the official solution,
please state that, and point out and explain any differences. If your solution
is correct, but very different from the official solution, write a brief
explanation to convince the grader that your solution is correct. If your
solution is partially correct, point out the parts that you got right, and
explain your mistakes.

\textbf{Answer:}

%%% PROBLEM 1(pE) ANSWER START %%%
My implementation uses standard binary search. The guess is the average between the current minimum and maximum values for the power. However, my implementation did not use PowExceeds. Instead, I simply computed the kth power of the the guess. This was done by multiplying the current middle with itself. In this part of the code, I have a small mistake. I should have multiplied the current nth power of the guess with the original guess value to give the $n + 1$st power of the guess.

If I had implemented the PowExceeds function used in the solution, then my code would have been able to check whether the guess was above or below the actual value in $\Theta(N^{log_2{6}})$ time. This would have made my code run in the correct asymptotic time. 
%%% PROBLEM 1(pE) ANSWER END %%%

\end{problems}

\end{document}
