\documentclass[psamsfonts]{amsart}

%-------Packages---------
\usepackage{amssymb,amsfonts}
\usepackage[all,arc]{xy}
\usepackage{enumerate}
\usepackage[margin=1in]{geometry}
\usepackage{amsthm}
\usepackage{theorem}
\usepackage{verbatim}
\usepackage{tikz}
\usepackage{bbold}
\usetikzlibrary{shapes,arrows}

\newenvironment{sol}{{\bfseries Solution:}}{\qedsymbol}
\newenvironment{prob}{{\bfseries Problem:}}
\newenvironment{def}{{\bfseries Definition:}}

\bibliographystyle{plain}

\voffset = -10pt
\headheight = 0pt
\topmargin = -20pt
\textheight = 690pt

%--------Meta Data: Fill in your info------
\title{6.857 \\
Network and Computer Security \\
Lecture 3}

\author{Lecturer: Ronald Rivest\\
Scribe: John Wang}

\begin{document}

\maketitle

\section{Growth of Cryptography - Continued}

\subsection{Zero Knowledge Proofs}

I can convince you that I know a solution to a hard problem while telling you nothing about my solution. I want to make sure you don't know anything about my solution.

Important for say, logging into a computer where you want to convince the computer that you know the secret key, but don't want to give the computer any information about the secret key.

\subsection{Micro Payments}

Micali and Rivest (2001). If you want to pay small amounts of money, how do you create a system for this?

Paying ten cents is equivalent to paying \$1 with probability 1/10. Uses pseudorandom digital signatures for verifiable fair dice. Neither the buyer or seller arranges the coin flips, there must be a cryptographically fair way of flipping the coin.

\subsection{Voting Systems}

There are new end-to-end cryptographic voting systems. All ballots are posted on the web and are encrypted, and the voters verify their votes are correct. Anyone can see the final tally. 

Uses cryptography for increasing transparency and verifiability. You don't want people selling their own vote. 

\subsection{Fully Homomorphic Encryption}

Can you compute on encrypted data, while keeping it encrypted? Given two encrypted data, are there ways to perform operations on them to create a new encrypted answer?

Craig Gentry in 2009 showed that you can do arbitrary computations on encrypted data based on the use of latticecs. Potential applications for cloud computing.

\section{Encryption and One Time Pads}

How do we get messages from Alice to Bob in a way that Eve can't listen in. There's some communication channel that Eve can listen to. Give Bob some key $K$, and Eve does not know $K$. 

Alice encrypts message $m$ and sends over $c \leftarrow enc(m)$ the encrpyted message over the channel.

Algorithms: 
\begin{itemize}
\item Key generation $K \leftarrow gen(\matbb{1}^\lambda)$ where $\lambda$ is key length. Here $\leftarrow$ denotes a randomized computation and $\mathbb{1}$ is a single bit. 
\item Encryption $c \leftarrow enc(K, m)$, where $c$ is the ciphertext that is sent over the channel.
\item Decryption $m = dec(K, c)$. Note that decryption is deterministic. 
\end{itemize}

\subsection{Notion of Encryption}

Ideas of what encryption is:
\begin{itemize}
\item Eve should not be able to produce $m$ from $c$. 
\item Can't identify any information about message and authenticate that the message is complete.
\item Shouldn't see patterns in messages.
\end{itemize}

The encryption game: Eve can't distinguish $enc(K,m_1)$ from $enc(K, m_2)$ where $m_1 \neq m_2$ and $|m_1| = |m_2|$ if $K$ is chosen randomly and Eve doesn't know $K$. Eve chooses $m_1, m_2$. Alice chooses one message $m_b$ and encrypts that $c \leftarrow enc(k, m_b)$. Eve tries to guess $b$ given ciphertext $c$.

\subsection{One Time Pad (Vernam 1917)}

Message $m$, ciphertext $c$, key $k$ are all $\lambda$ bit quantities. The key $k$ is also called the pad. 

Generation: Flip coin $\lambda$ times to get a pad $k$. 

Encryption: Bitwise exclusive-or, $c = m \oplus k$.

Decryption: Exlusive-or, $m = c \oplus k$. This works because $m = m \oplus k \oplus k$.

\subsection{Proof of Security}

\begin{thm}
One Time Pad is unconditionally/information-theoretically secure (Eve may have infinite computing power). 
\end{thm}
\begin{proof}
Let $P(m)$ be Eve's a priori probability of message $m$, or Eve's state of knowledge. Need to show $P(m | c) = P(m)$. 

We know that $|m| = |c| = |k| = \lambda$. We are assuming that $k$ is chosen randomly and the probability of any particular key is $P(k) = 2^{-\lambda}$. We know that $P(c|m) = 2^{-\lambda}$ = probability of $c$ given the message = probability that $k = m \oplus c$ = $2^{-\lambda}$.

\begin{eqnarray}
P(c) = \sum_{m} P(c|m) P(m) = \sum_{m} 2^{-\lambda) P(m) = 2^{-\lambda}
\end{eqnarray}

Where the last follows because $P(m) = 0$ unless $m$ is the actual message sent, in which case $P(m) = 1$. 

Now let us look at $P(m|c)$ which is the probability that Eve thinks message is $m$ having seen $c$.
\begin{eqnarray}
P(m|c) &=& \frac{P(c | m) P(m)}{P(c)} \\
&=& \frac{2^{-\lambda} P(m)}{2^{-\lambda}} \\
&=& P(m).
\end{eqnarray}

Thus, the ciphertext does not give Eve any information.
\end{proof}


\end{document}
