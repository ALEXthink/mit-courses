\documentclass[psamsfonts]{amsart}

%-------Packages---------
\usepackage{amssymb,amsfonts}
\usepackage{enumerate}
\usepackage[margin=1in]{geometry}
\usepackage{amsthm}
\usepackage{theorem}
\usepackage{verbatim}
\usepackage{tikz}
\usetikzlibrary{shapes,arrows}

\bibliographystyle{plain}

\voffset = -10pt
\headheight = 0pt
\topmargin = -20pt
\textheight = 690pt

%--------Meta Data: Fill in your info------
\title{6.857 \\
Network and Computer Security \\
Problem Set 4}

\author{John Wang}

\begin{document}

\maketitle

\section*{Problem 4.1}

\subsection*{Problem 4.1.a}

We will use an algorithm that uses a hash table in order to take the space and time requirements of the discrete log problem down to $\theta(\sqrt{k})$. This will rely on the assumption that hash tables have expected $\theta(1)$ read and write time (which is a legitimate assumption for reasonable hash functions).

Consider the following algorithm. First, we store $g^i$ for values of $i$ in the range $[0, m = 2^(k/2)]$. Then, for each $j$ in the same range, we check to see if $y (g^{-m})^j$ exists in the table. Formally, we have the following algorithm:

\begin{verbatim}

def discrete_log(p, g, y):
  m = 2^(k/2)
  g_results = {}
  for i in range(0, m):
    g_results[g^i % p] = i

  k = y
  for j in range(0, m):
    if k % p in g_results:
      i = g_results[k % p]
      return (i,j)
    else:
      k = k*inverse(g^m)
\end{verbatim}

Here we see that if any pair $(i,j)$ is such that $g^i (g^{2^(k/2)})^j \equiv g^{i + j (2^{k/2})} \equiv y \pmod{p}$, then we will find it with this algorithm. This is because all possible combinations of $i$ and $j$ are iterated over in this algorithm, with the first loop iterating over all possible values of $i$ and the second loop iterating over all possible values of $j$.

If any particular pair $(i,j)$ could be the correct pair for the discrete log problem with equal probability, then the expected number of pairs we must go through is $2^k$. However, we only need a runtime of $2^{k/2} (1 + 1/2)$ because we expect to finish halfway through the second loop. This is because we must complete the first loop in any given call to the \emph{discrete\_log} function, but each $j$ will have probability $1/m$ of hitting a correct result.

The total space is given by the number of items in the hash table, which is $2^{k/2}$.

\subsection{Problem 4.1.b}


\end{document}
